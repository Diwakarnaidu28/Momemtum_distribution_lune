%%%%%%%%%%%%%%%%%%%%%%%%%%%%%%%%%%%%%%%%%%%%%%%%%%%%%%%%%%%%%%%%%%%%%%
%%                                                                 %%
%% Please do not use \input{...} to include other tex files.       %%
%% Submit your LaTeX manuscript as one .tex document.              %%
%%                                                                 %%
%% All additional figures and files should be attached             %%
%% separately and not embedded in the \TeX\ document itself.       %%
%%                                                                 %%
%%%%%%%%%%%%%%%%%%%%%%%%%%%%%%%%%%%%%%%%%%%%%%%%%%%%%%%%%%%%%%%%%%%%%

%%\documentclass[referee,sn-basic]{sn-jnl}% referee option is meant for double line spacing
%%=======================================================%%
%% to print line numbers in the margin use lineno option %%
%%=======================================================%%

%%\documentclass[lineno,sn-basic]{sn-jnl}% Basic Springer Nature Reference Style/Chemistry Reference Style

%%======================================================%%
%% to compile with pdflatex/xelatex use pdflatex option %%
%%======================================================%%

%%\documentclass[pdflatex,sn-basic]{sn-jnl}% Basic Springer Nature Reference Style/Chemistry Reference Style


%%Note: the following reference styles support Namedate and Numbered referencing. By default the style follows the most common style. To switch between the options you can add or remove “Numbered” in the optional parenthesis. 
%%The option is available for: sn-basic.bst, sn-vancouver.bst, sn-chicago.bst, sn-mathphys.bst. %  
 
%%\documentclass[sn-nature]{sn-jnl}% Style for submisstotalions to Nature Portfolio journals
%%\documentclass[sn-basic]{sn-jnl}% Basic Springer Nature Reference Style/Chemistry Reference Style
\documentclass[sn-mathphys,Numbered, a4paper ,nocrop]{sn-jnl}% Math and Physical Sciences Reference Style
%%\documentclass[sn-aps]{sn-jnl}% American Physical Society (APS) Reference Style
%%\documentclass[sn-vancouver,Numbered]{sn-jnl}% Vancouver Reference Style
%%\documentclass[sn-apa]{sn-jnl}% APA Reference Style 
%%\documentclass[sn-chicago]{sn-jnl}% Chicago-based Humanities Reference Style
%%\documentclass[default]{sn-jnl}% Default
%%\documentclass[default,iicol]{sn-jnl}% Default with double column layout

%%%% Standard Packages
%%<additional latex packages if required can be included here>
\usepackage{amsmath, amssymb, amsfonts, physics, braket, hhline, mathtools, cancel, bigints,eufrak,geometry}
\geometry{total={210mm,297mm},bottom={40mm},}
\usepackage{pgfplots, subcaption, floatrow, footnote, adjustbox,float,fancyvrb}
\usepackage{graphicx, grffile, epsfig, listings,romannum, hyperref}
\usepackage{verbatim}
\usepackage{textcomp}
\usepackage{pdfpages}
\usepackage{accents}
\usepackage{tikz-cd}
\usepackage{multirow}%
\usepackage{amsthm}%
\usepackage{mathrsfs}%
\usepackage[title]{appendix}%
\usepackage{xcolor}%
\usepackage{textcomp}%
\usepackage{manyfoot}%
\usepackage{booktabs}%
\usepackage{algorithm}%
\usepackage{algorithmicx}%
\usepackage{algpseudocode}%
\pgfplotsset{compat=1.9}
\usetikzlibrary{shapes, arrows.meta, positioning, shapes.geometric}
\usepackage[capitalise]{cleveref}
%%%%
%%%%%%%%%%%%%%%%%%%%%%%%%%%%%%%%%
\DeclareMathOperator{\R}{\mathbb{R}}
\DeclareMathOperator{\C}{\mathbb{C}}
\DeclareMathOperator{\N}{\mathbb{N}}
\DeclareMathOperator{\Z}{\mathbb{Z}}
\DeclareMathOperator{\T}{\mathbb{T}}

\DeclareMathOperator{\QQ}{\mathcal{Q}}
\DeclareMathOperator{\HH}{\mathcal{H}}
\DeclareMathOperator{\LL}{\mathcal{L}}
\DeclareMathOperator{\KK}{\mathcal{K}}

\DeclareMathOperator{\SH}{\mathscr{H}}
\DeclareMathOperator{\Psis}{\Psi^*}
\newcommand{\bint}{\bigintssss}
\newcommand\Item[1][]{%
  \ifx\relax#1\relax  \item \else \item[#1] \fi
  \abovedisplayskip=0pt\abovedisplayshortskip=0pt~\vspace*{-\baselineskip}}
\newcommand{\ep}{\varepsilon}
\newcommand{\dg}{^\dagger}
\newcommand{\half}{\frac{1}{2}}
\newcommand{\eva}[1]{\left\langle #1 \right\rangle}
\newcommand{\bracket}[2]{\left\langle #1 | #2 \right\rangle}
\renewcommand{\det}[1]{\mathrm{det}\left( #1 \right)}
\newcommand{\del}[1]{\frac{\partial}{\partial #1}}
\newcommand{\fulld}[1]{\frac{d}{d #1}}
\newcommand{\fulldd}[2]{\frac{d #1}{d #2}}
\newcommand{\dell}[2]{\frac{\partial #1}{\partial #2}}
\newcommand{\delltwo}[2]{\frac{\partial^2 #1}{\partial #2 ^2}}  
\newcommand{\com}[1]{\left[ #1 \right]}
%%%%%%%%%%%%%%%%%%%%%%%%%%%%%%%%%%%%%%%%%%%%%%%%%%%
% THEOREMSTYLES
\theoremstyle{plain}
\newtheorem{theorem}{Theorem}[section]
\newtheorem{lemma}[theorem]{Lemma}
\newtheorem{corollary}[theorem]{Corollary}
\newtheorem{observation}[theorem]{Observation}
\newtheorem{proposition}[theorem]{Proposition}

\theoremstyle{definition}
\newtheorem{definition}[theorem]{Definition}
\newtheorem{problem}[theorem]{Problem}
\newtheorem{assumption}[theorem]{Assumption}
\newtheorem{example}[theorem]{Example}

\theoremstyle{remark}
\newtheorem{claim}[theorem]{Claim}
\newtheorem{remark}[theorem]{Remark}

% UNNUMBERED VERSIONS
\theoremstyle{plain}
\newtheorem*{theorem*}{Theorem}
\newtheorem*{lemma*}{Lemma}
\newtheorem*{corollary*}{Corollary}
\newtheorem*{proposition*}{Proposition}


\theoremstyle{definition}
\newtheorem*{definition*}{Definition}
\newtheorem*{problem*}{Problem}
\newtheorem*{assumption*}{Assumption}
\newtheorem*{example*}{Example}

\theoremstyle{remark}
\newtheorem*{claim*}{Claim}
\newtheorem*{remark*}{Remark}
%%\newtheorem{theorem}{Theorem}[section]% meant for sectionwise numbers
%% optional argument [theorem] produces theorem numbering sequence instead of independent numbers for Proposition
%%%%%%%%%%%%%%%%%%%%%%%%%%%%%%%%%%%%%%

\begin{document}

\title[Article Title]{Occupation Density}

%%=============================================================%%
%% Prefix	-> \pfx{Dr}
%% GivenName	-> \fnm{Joergen W.}
%% Particle	-> \spfx{van der} -> surname prefix
%% FamilyName	-> \sur{Ploeg}
%% Suffix	-> \sfx{IV}
%% NatureName	-> \tanm{Poet Laureate} -> Title after name
%% Degrees	-> \dgr{MSc, PhD}
%% \author*[1,2]{\pfx{Dr} \fnm{Joergen W.} \spfx{van der} \sur{Ploeg} \sfx{IV} \tanm{Poet Laureate} 
%%                 \dgr{MSc, PhD}}\email{iauthor@gmail.com}
%%=============================================================%%

%\author[1]{\fnm{Niels} \sur{Benedikter}}\email{niels.benedikter@unimi.it}

%\author[1]{\fnm{Diwakar} \sur{Naidu}}\email{diwakar.naidu@unimi.it}
%\equalcont{These authors contributed equally to this work.}

%\author[1,2]{\fnm{Third} \sur{Author}}\email{iiiauthor@gmail.com}
%\equalcont{These authors contributed equally to this work.}

%\affil[1]{\orgdiv{Department}, \orgname{Organization}, \orgaddress{\street{Street}, \city{City}, \postcode{100190}, \state{State}, \country{Country}}}

%\affil[2]{\orgdiv{Department}, \orgname{Organization}, \orgaddress{\street{Street}, \city{City}, \postcode{10587}, \state{State}, \country{Country}}}

%\affil[3]{\orgdiv{Department}, \orgname{Organization}, \orgaddress{\street{Street}, \city{City}, \postcode{610101}, \state{State}, \country{Country}}}

%%==================================%%
%% sample for unstructured abstract %%
%%==================================%%

\abstract{The }

%%================================%%
%% Sample for structured abstract %%
%%================================%%

% \abstract{\textbf{Purpose:} The abstract serves both as a general introduction to the topic and as a brief, non-technical summary of the main results and their implications. The abstract must not include subheadings (unless expressly permitted in the journal's Instructions to Authors), equations or citations. As a guide the abstract should not exceed 200 words. Most journals do not set a hard limit however authors are advised to check the author instructions for the journal they are submitting to.
% 
% \textbf{Methods:} The abstract serves both as a general introduction to the topic and as a brief, non-technical summary of the main results and their implications. The abstract must not include subheadings (unless expressly permitted in the journal's Instructions to Authors), equations or citations. As a guide the abstract should not exceed 200 words. Most journals do not set a hard limit however authors are advised to check the author instructions for the journal they are submitting to.
% 
% \textbf{Results:} The abstract serves both as a general introduction to the topic and as a brief, non-technical summary of the main results and their implications. The abstract must not include subheadings (unless expressly permitted in the journal's Instructions to Authors), equations or citations. As a guide the abstract should not exceed 200 words. Most journals do not set a hard limit however authors are advised to check the author instructions for the journal they are submitting to.
% 
% \textbf{Conclusion:} The abstract serves both as a general introduction to the topic and as a brief, non-technical summary of the main results and their implications. The abstract must not include subheadings (unless expressly permitted in the journal's Instructions to Authors), equations or citations. As a guide the abstract should not exceed 200 words. Most journals do not set a hard limit however authors are advised to check the author instructions for the journal they are submitting to.}

\keywords{keyword1, Keyword2, Keyword3, Keyword4}

%%\pacs[JEL Classification]{D8, H51}

%%\pacs[MSC Classification]{35A01, 65L10, 65L12, 65L20, 65L70}

\maketitle

\section{Introduction}\label{sec1}
We consider a quantum system of N spinless fermionic particles on $\mathbb{T}\coloneq [0,2\pi]^3$. The system is described by the Hamiltonian
\begin{equation}
    H = -\hbar^2\sum\limits_{j=1}^{N}\Delta_{x_j} + \lambda\!\!\!\sum\limits_{1\leq i < j \leq N } V(x_i - x_j)
\end{equation}
acting on the wave functions in the anti-symmetric tensor product $L^2_a(\T^{3N}) = \bigwedge_{i=1}^N L^2_a(\T^3)$.
We want to find the occupation density in the asymptotic limit when $N\rightarrow\infty$ in the \textit{mean-field scaling regime} i.e. we set
\begin{equation}
    \hbar\coloneq N^{-\frac{1}{3}}, \quad\text{and}\quad \lambda \coloneq N^{-1}
\end{equation}



Then we have
\begin{align}
    \eva{\Psi_{trial},n_q\Psi_{trial}} &= \eva{\Psi_{trial},a^*_qa_q\Psi_{trial}} 
\end{align}

%\section{Construction of Trial state}\label{sec:construction}


\section{Computations}\label{sec2}

Consider a trial state $\Psi_{trial}$ such that $\braket{\Psi_{trial},H\Psi_{trial}} = E_{HF} + E_{RPA}+ o(\hbar) $, where $E_{HF}$ is the Hartree-Fock energy and $E_{RPA}$ is the correlation energy from RPA.

We need to calculate $\braket{\Psi_{trial},a^*_la_l\Psi_{trial}},\, l\in\mathbb{Z}^3$. Here the trial state $\Psi_{trial}= Re^k\Omega$, where 
\begin{equation}
    R\Omega = \frac{1}{\sqrt{N!}}\text{det}\left(\frac{1}{(2\pi)^{3/2}}e^{ik_j\cdot x_i}\right)^N_{j,i=1}.
\end{equation}
is the Slater determinant of all plane waves with $N$ different momenta $k_j \in \Z^3$.
We have the Fermi ball i.e. states filling up all the momenta up to Fermi momentum as
\begin{equation}
    B_F\coloneq\left\{k\in \Z^3 : |k|\leq k_F\right\}
\end{equation}
for some $k_F>0$ and we define its complement as 
\begin{equation}
    B_F^c=\Z^3\backslash B_F
\end{equation}
Similarly we define a set of momenta which are outside the Fermi ball but are constrained to be a certain distance away from the Fermi ball as 
\begin{equation}
    L_k\coloneq \{p :p\in B_F^c \cap (B_F + k)\}
\end{equation}
with the following symmetry $L_{-k}=-L_k \quad\forall k \in \Z^3$.

We define the pair operators as\textcolor{red}{Write proper definition of the pair operators}
\begin{align}
    b_p(k) &= a_{p-k}a_{p}\\
    b^*_p(k) &= a^*_{p}a^*_{p-k}
\end{align}
for $p\in L_k$
\begin{lemma}[Quasi-Bosonic commutation relation]
   \begin{align}
       [b_{p}(k),b_{q}(\ell)] &= [b^*_{p}(k),b^*_{q}(\ell)] = 0\\
       [b_{p}(k),b^*_{q}(\ell)] &= \delta_{p,q}\delta_{k,\ell} + \epsilon_{p,q}(k,\ell),
   \end{align} where $\epsilon_{p,q}(k,\ell) = -\left(\delta_{p,q}a^*_{q-\ell}a_{p-k} + \delta_{p-k,q-\ell}a^*_{q}a_{p}\right)$ 
    \end{lemma}
\begin{proof}
    \begin{align}
        [b_{p}(k),b^*_{q}(\ell)] &= [a_{p-k}a_{p},a^*_{q}a^*_{q-\ell}]\nonumber\\
        &= a_{p-k}[a_p,a^*_{q}a^*_{q-\ell}] + [a_{p-k},a^*_{q}a^*_{q-\ell}]a_{p}\nonumber\\
        &= a_{p-k}\left\{a_p,a^*_{q}\right\}a^*_{q-\ell} - a_{p-k}a^*_{q}\{a_{p},a^*_{q-\ell}\}\nonumber \\ &+ \{a_{p-k},a^*_q\}a^*_{q-\ell}a_{p} - a^*_{q}\{a_{p-k},a^*_{q-\ell}\}a_{p}\nonumber\\
        &=\delta_{p,q}a_{p-k}a^*_{q-\ell} + \delta_{p-k,q-\ell}a^*_{q}a_{p}\nonumber\\
        &= \delta_{p,q}\delta_{k,\ell}-\left(\delta_{p,q}a^*_{q-\ell}a_{p-k} + \delta_{p-k,q-\ell}a^*_{q}a_{p}\right)
    \end{align}
Here, we denote the error term as $\epsilon_{p,q}(k,\ell) = -\left(\delta_{p,q}a^*_{q-\ell}a_{p-k} + \delta_{p-k,q-\ell}a^*_{q}a_{p}\right)$ with $\epsilon_{p,q}(k,k) = \epsilon^*_{q,p}(k,k) $ and $\epsilon_{p,p}(k,k)\geq 0$
\end{proof}
Also, we have the following identity
\begin{equation}
    [b^*_p(k), b_q(\ell)] = -[b_{p}(k),b^*_{q}(\ell)]^*
\end{equation}
with the effect of the complex conjugate seen only on the error term as above.  

Before we move on, we write some important commutation relations in order to facilitate further computations.
\begin{lemma}[Commutation relation between $a^\sharp_p$,\footnote{Here $\sharp = \{\;,*\} $} and $n_q$]\label{lem:coman}
    \begin{align}
        \com{n_q,a^*_p} &= \delta_{q.p}a^*_p\\
        \com{n_q,a_p} &= -\delta_{q.p}a_p
    \end{align}
\end{lemma} 
\begin{proof}
    \begin{align}
        \com{n_q,a^*_p} &= \com{a^*_qa_q,a^*_p}\nonumber\\
        &=a^*_qa_qa^*_p - a^*_pa^*_qa_q\nonumber\\
        &= a^*_q\delta_{qp}- a^*_qa^*_pa_q - a^*_pa^*_qa_q\nonumber\\
        &=\delta_{q,p}a^*_p
    \end{align}
    Here the second step follows from CAR for the fermionic creation and annihilation operators.
    
For the second commutation relation, we observe that 
    \begin{equation}
        \com{n_q,a_p}= -\com{n_q,a^*_p}^*.
    \end{equation}
Hence the commutation relation holds
\end{proof}
\begin{lemma}[Commutation relation between $b^\sharp_p$ and $n_q$]
    \begin{align}
        \com{n_q,b^*_p(k)} &= \left(\delta_{q.p}+\delta_{q.p-k}\right)b^*_p(k)\\
        \com{n_q,b_p(k)} &= -\left(\delta_{q.p}+\delta_{q.p-k}\right)b_p(k).
    \end{align}
\end{lemma} 
\begin{proof} We begin with the first commutation relation
    \begin{align}
        [n_q,b^*_p(k)] &= [n_q,a^*_pa^*_{p-k}]\\
        &=[n_q,a^*_p]a^*_{p-k}+a^*_p[n_q,a^*_{p-k}]\\
        &=\big(\delta_{q,p} +\delta_{q,p-k}\big)b^*_p(k).
    \end{align}
    It follows from the above Lemma \ref{lem:coman}. Similarly we observe
    \begin{equation}
         \com{n_q,b_p}= -\com{n_q,b^*_p}^*.
    \end{equation}
    And we attain the said relation for the second commutator.
\end{proof}

Consider a set of symmetric operator $K(\ell):\ell^2(L_\ell)\rightarrow \ell^2(L_\ell), \ell \in \Z^3_*$. Then we define the associated Bogoliubov kernel $\mathcal{K}:\HH_N\rightarrow\HH_N $ by
\begin{equation}
\mathcal{K} = \frac{1}{2}\sum\limits_{\ell\in \mathbb{Z}^3_*}\sum\limits_{r,s\in L_\ell}K(\ell)_{r,s}\left(b_r(\ell)b_{-s}(-\ell)-b^*_{-s}(-\ell)b^*_{r}(\ell)\right)
\end{equation}
Next, we define the Bogoliubov transformation $T = e^{\mathcal{K}} $ which is a unitary due the fact that $\mathcal{K}$ is anti-unitary i.e. $\mathcal{K}=-\mathcal{K}^* $.
%\begin{lemma}[Commutator between $\mathcal{K} $ and number operators]
%    For $q\in \Z^3_*$,
%    \begin{align}
%        [n_q,\mathcal{K}]&= \half\sum\limits_{\ell \in \Z^3_*}\sum\limits_{r,s \in L_{\ell}}K(\ell)_{r,s} \bigg((-1)\left(\delta_{q.r}+\delta_{q.r-\ell}+\delta_{q,-s}+\delta_{q,-s+\ell}\right)\\&\times\left(b_r(\ell)b_{-s}(-\ell)+b^*_{-s}(-\ell)b^*_{r}(\ell)\right)\bigg)\\
%        [n_{-q},\mathcal{K}]&=\half\sum\limits_{\ell \in \Z^3_*}\sum\limits_{r,s \in L_{\ell}}K(\ell)_{r,s} \bigg((-1)\left(\delta_{q.-r}+\delta_{q.-r+\ell}+\delta_{q,s}+\delta_{q,s-\ell}\right)\\&\times\left(b_r(\ell)b_{-s}(-\ell)+b^*_{-s}(-\ell)b^*_{r}(\ell)\right)\bigg)
%    \end{align}
%\end{lemma}

\begin{lemma}[Commutator between $\mathcal{K} $ and pair operators]
\begin{align}
    [b^*_p(k),\mathcal{K}] &=-\sum\limits_{s\in L_{k}}K(k)_{p,s}b_{-s}(-k) + \mathcal{E}_{p}(k)\label{eq:13} \\
    [b_p(k),\mathcal{K}] &=-\sum\limits_{s\in L_{k}}K(k)_{p,s}b^*_{-s}(-k) + \left(\mathcal{E}_{p}(k)\right)^*\label{eq:14},
\end{align}
    where, 
\begin{equation}
    \mathcal{E}_{p}(k) = -\frac{1}{2}\sum\limits_{l\in \mathbb{Z}^3_*}\sum\limits_{r,s\in L_l}K(\ell)_{r,s}\left\{\epsilon_{r,p}(\ell,k),b_{-s}(-\ell)\right\} 
\end{equation}
\end{lemma}
\begin{proof}
We start with the first commutation relation.
   \begin{alignat}{2}
      [b^*_p(k),\mathcal{K}]&= \left[b^*_p(k),\half\sum\limits_{\ell\in \mathbb{Z}^3_*}\sum\limits_{r,s\in L_\ell}K(\ell)_{r,s}\left(b_r(\ell)b_{-s}(-\ell)-b^*_{-s}(-\ell)b^*_{r}(\ell)\right)\right]\nonumber\\  
      &=\half\sum\limits_{\ell\in \mathbb{Z}^3_*}\sum\limits_{r,s\in L_\ell}K(\ell)_{r,s}\left[b^*_p(k),b_r(\ell)b_{-s}(-\ell)\right]\nonumber\\
      &= \half\sum\limits_{\ell\in \mathbb{Z}^3_*}\sum\limits_{r,s\in L_\ell}K(\ell)_{r,s}\left(\left[b^*_p(k),b_r(\ell)\right]b_{-s}(-\ell) +b_{r}(\ell)\left[b^*_p(k),b_{-s}(-\ell)\right]\right)\nonumber\\
      &=\half\sum\limits_{\ell\in \mathbb{Z}^3_*}\sum\limits_{r,s\in L_\ell}K(\ell)_{r,s}\Big(\big(-\delta_{p,r}\delta_{k,\ell} -\epsilon_{r,p}(\ell,k)\big)b_{-s}(-\ell) +b_{r}(\ell)\big(-\delta_{p,-s}\delta_{k,-\ell} -\epsilon_{-s,p}(-\ell,k)\big)\Big)\nonumber\\
      &=\begin{aligned}[t]
          &-\half\sum\limits_{\ell\in \mathbb{Z}^3_*}\sum\limits_{r,s\in L_\ell}K(\ell)_{r,s}\big(\delta_{p,r}\delta_{k,\ell}\big)b_{-s}(-\ell) - \half\sum\limits_{\ell\in \mathbb{Z}^3_*}\sum\limits_{r,s\in L_\ell}K(\ell)_{r,s}\big(\epsilon_{r,p}(\ell,k)b_{-s}(-\ell)\big) \\
      &- \half\sum\limits_{\ell\in \mathbb{Z}^3_*}\sum\limits_{r,s\in L_\ell}K(\ell)_{r,s}b_{r}(\ell)\big(\delta_{p,-s}\delta_{k,-\ell}\big)\, - \half\sum\limits_{\ell\in \mathbb{Z}^3_*}\sum\limits_{r,s\in L_\ell}K(\ell)_{r,s}\big(b_{r}(\ell)\epsilon_{-s,p}(-\ell,k)\big)
      \end{aligned}\nonumber\\
      &=\begin{aligned}[t]
          &-\half\sum\limits_{s\in L_k}K(k)_{p,s}b_{-s}(-k) - \half\sum\limits_{r\in L_{-k}}K(-k)_{r,-p}b_{r}(-k) \\
      &- \half\sum\limits_{\ell\in \mathbb{Z}^3_*}\sum\limits_{r,s\in L_\ell}K(\ell)_{r,s}\big(\epsilon_{r,p}(\ell,k)b_{-s}(-\ell)\big) - \half\sum\limits_{\ell\in \mathbb{Z}^3_*}\sum\limits_{r,s\in L_\ell}K(\ell)_{r,s}\big(b_{r}(\ell)\epsilon_{-s,p}(-\ell,k)\big).\label{eq:opencomKb}
      \end{aligned}
   \end{alignat}
   Consider the second summand, we know that $L_{-k}=-L_k$,
   then we identify $r$ with $-s$ and we have
   \begin{equation}\label{eq:2ndsummand}
        - \half\sum\limits_{-s\in -L_{k}}K(-k)_{-s,-p}b_{-s}(-k) =- \half\sum\limits_{s\in L_{k}}K(k)_{s,p}b_{-s}(-k) .
   \end{equation}
   Now, consider the fourth summand, first we exchange $r$ and $s$ and arrive at
   \begin{equation}\label{eq:beforeflip}
       - \half\sum\limits_{\ell\in \mathbb{Z}^3_*}\sum\limits_{r,s\in L_\ell}K(\ell)_{r,s}\big(b_{s}(\ell)\epsilon_{-r,p}(-\ell,k)\big).
   \end{equation}
   Second, we reflect all the summed over momenta (i.e. $\ell\rightarrow-\ell, r\rightarrow-r, s\rightarrow-s$) which provides us
   \begin{equation}\label{eq:4thsummand}
       (\ref{eq:beforeflip}) = - \half\sum\limits_{\ell\in \mathbb{Z}^3_*}\sum\limits_{r,s\in L_\ell}K(\ell)_{r,s}\big(b_{-s}(-\ell)\epsilon_{r,p}(\ell,k)\big).
   \end{equation}
   Then substituting (\ref{eq:2ndsummand}) and (\ref{eq:4thsummand}) in (\ref{eq:opencomKb}), we get
   \begin{align}
       (\ref{eq:opencomKb})=&-\sum\limits_{s\in L_k}K(k)_{p,s}b_{-s}(-k)  \\
      &- \half\sum\limits_{\ell\in \mathbb{Z}^3_*}\sum\limits_{r,s\in L_\ell}K(\ell)_{r,s}\big(\epsilon_{r,p}(\ell,k)b_{-s}(-\ell)+b_{-s}(-\ell)\epsilon_{r,p}(\ell,k)\big)\label{eq:errbK}
   \end{align}
   Here, we observe (\ref{eq:errbK}) $=  \mathcal{E}_{p}(k) $.
\end{proof}
Before we begin the evaluation, we define

\textbf{Symmetry transformation} Symmetry transformation is a unitary transformation $\mathfrak{R}:\mathcal{F}\rightarrow \mathcal{F}$ defined by its action as
\begin{equation}
    \mathfrak{R}: a^*_{k_1}\ldots a^*_{k_n}\Omega \mapsto a^*_{-k_1}\ldots a^*_{-k_n}\Omega 
\end{equation}
while leaving the vacuum state invariant.

\begin{lemma}\label{lem:symtransformation}
    For the symmetry transformation $\mathfrak{R}$ and the almost bosonic Bogoliubov transformation $T$, we have
    \begin{equation}
        \mathfrak{R}T\Omega = T\Omega
    \end{equation}
\end{lemma}
%\begin{proof}\textcolor{red}{to be filled in}
    %We begin with
    %\begin{align}
     %   \eva{\mathfrak{R}T\Omega, a^*_qa_q\mathfrak{R}T\Omega} &= \eva{T\Omega, \mathfrak{R}^*a^*_qa_q\mathfrak{R}T\Omega}\\
    %\end{align}
%\end{proof}
From this lemma we observe that 
\begin{equation}
    \eva{T\Omega\, a^*_qa_qT\Omega} = \eva{T\Omega\, a^*_{-q}a_{-q}T\Omega}
\end{equation}
And hence motivated by Lemma \ref{lem:symtransformation}, we evaluate $\eva{\Omega, T_1^*\half\left(n_q+n_{-q}\right)T_1\Omega}$.\newline 

Next we define the quadratic operator as 
\begin{definition}
For $l \in \Z^3_*$
\begin{align} 
    Q_1(A(\ell))&\coloneq  -\sum\limits_{\ell \in \Z^3_*}\sum\limits_{r,s \in L_{\ell}}A(\ell)_{r,s} \left(b^*_r(\ell)b_{s}(\ell)+b_{r}(\ell)b^*_{s}(\ell)\right)\label{eq:Q1}\\ 
    Q_2(A(\ell))&\coloneq  \sum\limits_{\ell \in \Z^3_*}\sum\limits_{r,s \in L_{\ell}}A(\ell)_{r,s} \left(b_r(\ell)b_{-s}(-\ell)+b^*_{-s}(-\ell)b^*_{r}(\ell)\right)\label{eq:Q2}
\end{align}
    
\end{definition}

 
\textbf{Evaluation of the expectation value}
\begin{lemma}
For $q \in B_F^c$, we define the projection operator, projecting to momentum $q$ and $-q$, $ (\Delta^q)_{m,s}\coloneq -\half(\delta_{m,q}\delta_{m,s}+\delta_{m,-q}\delta_{m.s})$ and we get
\begin{equation}\label{eq: mainexp}
    \eva{\Omega, T_1^*\half\left(n_q+n_{-q}\right)T_1\Omega} =  \half\bint\limits_0^1 \!\!d\lambda
     \bigg<\Omega,T_\lambda^* Q_2 \Big(\big\{K(\ell),\Delta^q\big\} \Big)T_\lambda\Omega\bigg>
\end{equation}   
\end{lemma}
\begin{proof}
We start by applying Duhamel's formula to RHS of (\ref{eq: mainexp}) and we have
\begin{align}
    &\half\left(\eva{\Omega,\left(n_q+n_{-q}\right)\Omega} + \bint\limits_0^1 d\lambda  \left(\fulld{\lambda}\eva{\Omega, T_\lambda^*\left(n_q+n_{-q}\right)T_\lambda\Omega} \right)\right)\\
    &=\half\bint\limits_0^1 d\lambda  \left(\eva{\Omega, T_\lambda^*(-\mathcal{K})\left(n_q+n_{-q}\right)T_\lambda+T_\lambda^*\left(n_q+n_{-q}\right)(\mathcal{K})T_\lambda\Omega} \right)\\
    &= \half\bint\limits_0^1 d\lambda  \eva{\Omega, T_\lambda^*[\left(n_q+n_{-q}\right),\mathcal{K}]T_\lambda\Omega}\label{eq:halfexp}.
\end{align}
Next using the definition of $\mathcal{K}$, we write the expression for the commutator.
\begin{align}
    \left[n_q,\mathcal{K}\right]= \half\sum\limits_{\ell \in \Z^3_*}\sum\limits_{r,s \in L_{\ell}}K(\ell)_{r,s}&\left[a^*_q a_q, \left(b_r(\ell)b_{-s}(-\ell)-b^*_{-s}(-\ell)b^*_{r}(\ell)\right)\right]\\
    = \half\sum\limits_{\ell \in \Z^3_*}\sum\limits_{r,s \in L_{\ell}}K(\ell)_{r,s}
         &\bigg(\left[a^*_q a_q, b_r(\ell)\right]b_{-s}(-\ell) + b_{r}(\ell)\left[a^*_q a_q, b_{-s}(-\ell)\right]\\  &- \left[a^*_q a_q,b^*_{-s}(-\ell)\right]b^*_{r}(\ell) -b^*_{-s}(-\ell)\left[a^*_q a_q,b^*_{r}(\ell)\right]\bigg)\\
    =\half\sum\limits_{\ell \in \Z^3_*}\sum\limits_{r,s \in L_{\ell}}K(\ell)_{r,s} &\bigg((-1)\left(\delta_{q.r}+\delta_{q.r-\ell}+\delta_{q,-s}+\delta_{q,-s+\ell}\right)\\&\times\left(b_r(\ell)b_{-s}(-\ell)+b^*_{-s}(-\ell)b^*_{r}(\ell)\right)\bigg)
\end{align}    
 Now, since $q \in B_F^c$, $\delta_{q.r-\ell}=\delta_{q,-s+\ell}=0$, we have
 \begin{equation}\label{eq:nqcommuteK}
     \left[n_q,\mathcal{K}\right]= \half\sum\limits_{\ell \in \Z^3_*}\sum\limits_{r,s \in L_{\ell}} \!\! K(\ell)_{r,s} \bigg(\!(-1)\!\left(\delta_{q.r}+\delta_{q,-s}\right)\!\left(b_r(\ell)b_{-s}(-\ell)+b^*_{-s}(-\ell)b^*_{r}(\ell)\right)\!\!\bigg).
 \end{equation}
Similarly for $\left[n_q,\mathcal{K}\right]$, we have
\begin{equation}\label{eq:n-qcommuteK}
    \left[n_{-q},\mathcal{K}\right]= \half\sum\limits_{\ell \in \Z^3_*}\sum\limits_{r,s \in L_{\ell}} \!\! K(\ell)_{r,s} \bigg(\!(-1)\!\left(\delta_{-q.r}+\delta_{-q,-s}\right)\!\left(b_r(\ell)b_{-s}(-\ell)+b^*_{-s}(-\ell)b^*_{r}(\ell)\right)\!\!\bigg).
\end{equation}
Next we substitute commutators (\ref{eq:nqcommuteK}) and (\ref{eq:n-qcommuteK}) in (\ref{eq:halfexp}),
\begin{alignat}{2}
    (\ref{eq:halfexp}) &= \half\bint\limits_0^1 d\lambda\begin{aligned}[t]
     \Bigg<\Omega,T_\lambda^*\bigg( \half\sum\limits_{\ell \in \Z^3_*}\sum\limits_{r,s \in L_{\ell}} K(\ell)_{r,s} &\Big((-1)(\delta_{q.r}+\delta_{q,-s}+\delta_{-q.r}+\delta_{-q,-s})\\ &   \times(b_r(\ell)b_{-s}(-\ell)+b^*_{-s}(-\ell)b^*_{r}(\ell))\Big)\bigg)T_\lambda\Omega\Bigg> 
    \end{aligned}\nonumber\\
    &= \half\bint\limits_0^1 \!\!d\lambda\begin{aligned}[t]
     \Bigg<\Omega,T_\lambda^*\bigg( \sum\limits_{\ell \in \Z^3_*}\sum\limits_{r,s \in L_{\ell}} (-\half)&\Big(\underbrace{K(\ell)_{r,s} (\delta_{q.r}+\delta_{q,-s}+\delta_{-q.r}+\delta_{-q,-s})}_{\text{interpret as matrix product}}\\ &   \times(b_r(\ell)b_{-s}(-\ell)+b^*_{-s}(-\ell)b^*_{r}(\ell))\Big)\bigg)T_\lambda\Omega\Bigg> 
    \end{aligned}\\
    &= \half\bint\limits_0^1 \!\!d\lambda\begin{aligned}[t]
     &\Bigg<\Omega,T_\lambda^*\bigg( \sum\limits_{\ell \in \Z^3_*}\sum\limits_{r,s \in L_{\ell}}\Big(K(\ell)_{r,m} (-\half)\underbrace{(\delta_{m,q}\delta_{m,s}+\delta_{m,-q}\delta_{m.s})}_{\text{(a)}}\\ &  + (-\half)\underbrace{(\delta_{r,q}\delta_{r,m}+\delta_{r,-q}\delta_{r.m})}_{\text{(b)}}K(\ell)_{m,s} \Big)(b_r(\ell)b_{-s}(-\ell)+b^*_{-s}(-\ell)b^*_{r}(\ell))\bigg)T_\lambda\Omega\Bigg>\label{eq:momentumfix} 
    \end{aligned}
\end{alignat}
Next, we observe that (a) and (b) are projection of a momentum ($r$ or $s \in L_\ell$) to momentum $q$ and $-q$.
We then arrive at
\begin{alignat}{2}
    (\ref{eq:momentumfix}) &=  \half\bint\limits_0^1 d\lambda\begin{aligned}[t]
     \bigg<\Omega,T_\lambda^*\bigg( \sum\limits_{\ell \in \Z^3_*}\sum\limits_{r,s \in L_{\ell}}&\Big(K(\ell)_{r,m}\Delta^q_{m,s}  +\Delta^q_{r,m} K(\ell)_{m,s} \Big)\\&(b_r(\ell)b_{-s}(-\ell)+b^*_{-s}(-\ell)b^*_{r}(\ell))\bigg)T_\lambda\Omega\bigg>
    \end{aligned}\\
    &= \half\bint\limits_0^1 \!\!d\lambda\begin{aligned}[t]
     \bigg<\Omega,T_\lambda^*\bigg( \sum\limits_{\ell \in \Z^3_*}\sum\limits_{r,s \in L_{\ell}}&\Big(K(\ell)_{r,m}\Delta^q_{m,s}  +\Delta^q_{r,m} K(\ell)_{m,s} \Big)\\&(b_r(\ell)b_{-s}(-\ell)+b^*_{-s}(-\ell)b^*_{r}(\ell))\bigg)T_\lambda\Omega\bigg> 
    \end{aligned}\\
    &= \half\bint\limits_0^1 \!\!d\lambda
     \bigg<\Omega,T_\lambda^*\bigg( \sum\limits_{\ell \in \Z^3_*}\sum\limits_{r,s \in L_{\ell}}\Big\{K(\ell),\Delta^q\Big\}_{r,s}(b_r(\ell)b_{-s}(-\ell)+b^*_{-s}(-\ell)b^*_{r}(\ell))\bigg)T_\lambda\Omega\bigg>\label{eq:anticomKDelta} 
\end{alignat}
Using the definition of $Q_2$, \eqref{eq:Q2}, we arrive at 
\begin{equation}
    (\ref{eq:anticomKDelta}) = \half\bint\limits_0^1 \!\!d\lambda
     \bigg<\Omega,T_\lambda^* Q_2 \Big(\big\{K(\ell),\Delta^q\big\} \Big)T_\lambda\Omega\bigg>
\end{equation} \end{proof}
\begin{lemma}[Commutator between $\mathcal{K} $ and $Q_1$]
    \begin{equation}
        [ Q_1(A(\ell)),\mathcal{K}] = Q_2(\{A(\ell),K(\ell)\}) + E_{Q_1}(A(\ell))
    \end{equation}
 where,
 \begin{equation}
     E_{Q_1}(A(\ell))=-\sum\limits_{\ell \in \Z^3_*}\sum\limits_{r,s \in L_{\ell}}A(\ell)_{r,s}\Big(\big\{\mathcal{E}^*_{r}(\ell),b^*_{s}(\ell)\big\} + \big\{\mathcal{E}_{r}(\ell),b_{s}(\ell)\big\}\Big). 
 \end{equation}
\end{lemma}
\begin{proof}We begin with $[ Q_1(A(\ell)),\mathcal{K}]$.
    \begin{alignat}{2}
        [ Q_1(A(\ell)),\mathcal{K}] &= -\left[\sum\limits_{\ell \in \Z^3_*}\sum\limits_{r,s \in L_{\ell}}A(\ell)_{r,s}\Big(b^*_{r}(\ell)b_{s}(\ell) + b_{r}(\ell)b^*_{s}(\ell)\Big),\mathcal{K}\right]\\
        &=-\sum\limits_{\ell \in \Z^3_*}\sum\limits_{r,s \in L_{\ell}}A(\ell)_{r,s}\left[\Big(b^*_{r}(\ell)b_{s}(\ell) + b_{r}(\ell)b^*_{s}(\ell)\Big),\mathcal{K}\right]\\
        &=-\sum\limits_{\ell \in \Z^3_*}\sum\limits_{r,s \in L_{\ell}}A(\ell)_{r,s}\begin{aligned}[t]
            \Big(&b^*_{r}(\ell)\left[b_{s}(\ell),\mathcal{K}\right] +\left[b^*_{r}(\ell),\mathcal{K}\right]b_{s}(\ell)\\ + &b_{r}(\ell)\left[b^*_{s}(\ell),\mathcal{K}\right]+ \left[b_{r}(\ell),\mathcal{K}\right]b^*_{s}(\ell)\Big)
        \end{aligned}\label{eq:Q1K1}
    \end{alignat}
    Now we use the commutation relation \eqref{eq:13} and \eqref{eq:14} to get
\begin{alignat}{2}
    (\ref{eq:Q2K1})&=-\sum\limits_{\ell \in \Z^3_*}\sum\limits_{r,s \in L_{\ell}}A(\ell)_{r,s}\begin{aligned}[t]
        &\Bigg(b^*_{r}(\ell)\Bigg(-\sum\limits_{s'\in L_{\ell}}K(\ell)_{s,s'}b^*_{-s'}(-\ell) + \mathcal{E}^*_{s}(\ell)\Bigg)\\ &+ \left(-\sum\limits_{s'\in L_{\ell}}K(\ell)_{r,s'}b_{-s'}(-\ell) + \mathcal{E}_{r}(\ell)\right)b_{s}(\ell)\\&+ b_{r}(\ell)\left(-\sum\limits_{s'\in L_{\ell}}K(\ell)_{s,s'}b_{-s'}(-\ell) + \mathcal{E}_{s}(\ell)\right)\\ &+ \left(-\sum\limits_{s'\in L_{\ell}}K(\ell)_{r,s'}b^*_{-s'}(-\ell) + \mathcal{E}^*_{r}(\ell)\right)b^*_{s}(\ell) \Bigg)        
    \end{aligned}\\
    &=\sum\limits_{\ell \in \Z^3_*}\sum\limits_{r,s,s' \in L_{\ell}}A(\ell)_{r,s}\begin{aligned}[t]
        \Big(&K(\ell)_{s,s'}b^*_{r}(\ell)b^*_{-s'}(-\ell) -b^*_{r}(\ell) \mathcal{E}^*_{s}(\ell)\\ + &K(\ell)_{r,s'}b_{-s'}(-\ell)b_{s}(\ell) - \mathcal{E}_{r}(\ell)b_{s}(\ell)\\+& K(\ell)_{s,s'}b_{r}(\ell)b_{-s'}(-\ell) - b_{r}(\ell)\mathcal{E}_{s}(\ell)\\ + &K(\ell)_{r,s'}b^*_{-s'}(-\ell)b^*_{s}(\ell) - \mathcal{E}^*_{r}(\ell)b^*_{s}(\ell) \Big) 
    \end{aligned}\\
    &=\sum\limits_{\ell \in \Z^3_*}\sum\limits_{r,s,s' \in L_{\ell}}A(\ell)_{r,s}\begin{aligned}[t]
        \Big(&K(\ell)_{s,s'}\big(b^*_{r}(\ell)b^*_{-s'}(-\ell)+b_{r}(\ell)b_{-s'}(-\ell)\big)\\ + &K(\ell)_{r,s'}\big(b_{-s'}(-\ell)b_{s}(\ell)+b^*_{-s'}(-\ell)b^*_{s}(\ell) \big)\Big) 
    \end{aligned}\nonumber\\
    &-\sum\limits_{\ell \in \Z^3_*}\sum\limits_{r,s \in L_{\ell}}A(\ell)_{r,s}\Big(b^*_{r}(\ell)\mathcal{E}^*_{s}(\ell) + \mathcal{E}_{r}(\ell)b_{s}(\ell) + b_{r}(\ell)\mathcal{E}_{s}(\ell) + \mathcal{E}^*_{r}(\ell)b^*_{s}(\ell)\Big).\label{eq:Q1Kerr_not_con}
\end{alignat}
Now we represent the second sum in (\ref{eq:Q1Kerr_not_con}) as $E_{Q_1}(A(\ell))$. Furthermore, we exchange $r$ and $s$ in first and third term of second sum in (\ref{eq:Q1Kerr_not_con}) and we have
\begin{align}
    E_{Q_1}(A(\ell))&= -\sum\limits_{\ell \in \Z^3_*}\sum\limits_{r,s \in L_{\ell}}A(\ell)_{r,s}\Big(b^*_{s}(\ell)\mathcal{E}^*_{r}(\ell) + \mathcal{E}_{r}(\ell)b_{s}(\ell) + b_{s}(\ell)\mathcal{E}_{r}(\ell) + \mathcal{E}^*_{r}(\ell)b^*_{s}(\ell)\Big)\\
    &= -\sum\limits_{\ell \in \Z^3_*}\sum\limits_{r,s \in L_{\ell}}A(\ell)_{r,s}\Big(\big\{\mathcal{E}^*_{r}(\ell),b^*_{s}(\ell)\big\} + \big\{\mathcal{E}_{r}(\ell),b_{s}(\ell)\big\}\Big).
\end{align}
Continuing with (\ref{eq:Q1Kerr_not_con}) while having the error $E_{Q_1}(A(\ell))$.
\begin{align}
    (\ref{eq:Q1Kerr_not_con}) = \sum\limits_{\ell \in \Z^3_*}\sum\limits_{r,s,s' \in L_{\ell}}A(\ell)_{r,s}
        \Big(&K(\ell)_{s,s'}\big(b^*_{r}(\ell)b^*_{-s'}(-\ell)+b_{r}(\ell)b_{-s'}(-\ell)\big)\nonumber\\ + &K(\ell)_{r,s'}\big(b_{-s'}(-\ell)b_{s}(\ell)+b^*_{-s'}(-\ell)b^*_{s}(\ell) \big)\Big) + E_{Q_1}(A(\ell))\label{eq:Q1Knoiden}
\end{align}
Then we do a sequence of identifications on the second term, first we exchange $s$ and $s'$ 
\begin{equation}
    \sum\limits_{\ell \in \Z^3_*}\sum\limits_{r,s,s' \in L_{\ell}}A(\ell)_{r,s'}K(\ell)_{r,s}\big(b_{-s}(-\ell)b_{s'}(\ell)+b^*_{-s}(-\ell)b^*_{s'}(\ell) \big)\Big).
\end{equation}
Next we exchange $r$ and $s$ and arrive at
\begin{equation}
    \sum\limits_{\ell \in \Z^3_*}\sum\limits_{r,s,s' \in L_{\ell}}A(\ell)_{s,s'}K(\ell)_{s,r}\big(b_{-r}(-\ell)b_{s'}(\ell)+b^*_{-r}(-\ell)b^*_{s'}(\ell) \big)\Big).
\end{equation}
Finally we reflect all the momenta (i.e. $\ell\rightarrow -\ell,r\rightarrow -r,s\rightarrow -s,s'\rightarrow -s'$) and it gives us
\begin{equation}\label{eq:Q1kalliden}
    \sum\limits_{\ell \in \Z^3_*}\sum\limits_{r,s,s' \in L_{\ell}}A(\ell)_{s,s'}K(\ell)_{s,r}\big(b_{r}(\ell)b_{-s'}(-\ell)+b^*_{r}(\ell)b^*_{-s'}(-\ell) \big)\Big).
\end{equation}
Then substituting (\ref{eq:Q1kalliden}) in (\ref{eq:Q1Knoiden}) and interpreting the two terms as a matrix product, we arrive at
\begin{align}
    (\ref{eq:Q1Knoiden}) &= \sum\limits_{\ell \in \Z^3_*}\sum\limits_{r,s,s' \in L_{\ell}}\big\{A(\ell), K(\ell)\big\}_{r,s}\big(b_{r}(\ell)b_{-s}(-\ell)+b^*_{r}(\ell)b^*_{-s}(-\ell)  \big) + E_{Q_1}(A(\ell))\\
    &= Q_2\left(\big\{A(\ell), K(\ell)\big\}\right) + E_{Q_1}(A(\ell)).
\end{align}
\end{proof}

\begin{lemma}[Commutator between $\mathcal{K} $ and $Q_2$]
\begin{equation}
    [ Q_2(A(\ell)),\mathcal{K}] = Q_1(\{A(\ell),K(\ell)\}) + E_{Q_2}(A(\ell))
\end{equation}
 where,
\begin{equation}
     E_{Q_2}(A(\ell))=\sum\limits_{\ell \in \Z^3_*}\sum\limits_{r,s \in L_{\ell}}A(\ell)_{r,s}\Big(\big\{\mathcal{E}^*_{r}(\ell), b_{-s}(-\ell)\big\} + \big\{\mathcal{E}_r(l), b^*_{-s}(-l)\big\}\Big)
\end{equation}
\end{lemma}
\begin{proof}
We begin with $[ Q_2(A(\ell)),\mathcal{K}]$.
\begin{alignat}{2}
    [Q_2(A(\ell)),\mathcal{K}] &= \left[\sum\limits_{\ell \in \Z^3_*}\sum\limits_{r,s \in L_{\ell}}A(\ell)_{r,s}\Big(b_{r}(\ell)b_{-s}(-\ell) + b^*_{-s}(-\ell)b^*_{r}(\ell)\Big),\mathcal{K}\right]\\
    &= \sum\limits_{\ell \in \Z^3_*}\sum\limits_{r,s \in L_{\ell}}A(\ell)_{r,s}\left[\Big(b_{r}(\ell)b_{-s}(-\ell) + b^*_{-s}(-\ell)b^*_{r}(\ell)\Big),\mathcal{K}\right]\\
    &= \sum\limits_{\ell \in \Z^3_*}\sum\limits_{r,s \in L_{\ell}}A(\ell)_{r,s}\begin{aligned}[t]
        &\Big(b_{r}(\ell)[b_{-s}(-\ell),\mathcal{K}] + [b_{r}(\ell),\mathcal{K}]b_{-s}(-\ell)\\&+ b^*_{-s}(-\ell)[b^*_{r}(\ell),\mathcal{K}] + [b^*_{-s}(-\ell),\mathcal{K}]b^*_{r}(\ell) \Big)
    \end{aligned}\label{eq:Q2K1}
\end{alignat}
Now we use the commutation relation \eqref{eq:13} and \eqref{eq:14} to get
\begin{equation}
    (\ref{eq:Q2K1})=\sum\limits_{\ell \in \Z^3_*}\sum\limits_{r,s \in L_{\ell}}A(\ell)_{r,s}\begin{aligned}[t]
        &\Bigg(b_{r}(\ell)\Bigg(-\sum\limits_{s'\in L_{-\ell}}K(-\ell)_{-s,s'}b^*_{-s'}(\ell) + \mathcal{E}^*_{-s}(-\ell)\Bigg)\\ &+ \left(-\sum\limits_{s'\in L_{\ell}}K(\ell)_{r,s'}b^*_{-s'}(-\ell) + \mathcal{E}^*_{r}(\ell)\right)b_{-s}(-\ell)\\&+ b^*_{-s}(-\ell)\left(-\sum\limits_{s'\in L_{\ell}}K(\ell)_{r,s'}b_{-s'}(-\ell) + \mathcal{E}_{r}(\ell)\right)\\ &+ \left(-\sum\limits_{s'\in L_{-\ell}}K(-\ell)_{-s,s'}b_{-s'}(\ell) + \mathcal{E}_{-s}(-\ell)\right)b^*_{r}(\ell) \Bigg)
    \end{aligned}
\end{equation}
Next we do the identification $s'\rightarrow -s'$ and then use the symmetry $K(\ell)_{r,s} = K(-\ell)_{-r,-s}$ in the first and fourth term in order to bring all the sum over the new index $s'$ to the same lune
\begin{alignat}{2}
    &-\sum\limits_{\ell \in \Z^3_*}\sum\limits_{r,s,s' \in L_{\ell}}A(\ell)_{r,s}\begin{aligned}[t]
        &\Bigg(b_{r}(\ell)\Big(K(\ell)_{s,s'}b^*_{s'}(\ell) - \mathcal{E}^*_{-s}(-\ell)\Big)\\ &+ \Big(K(\ell)_{r,s'}b^*_{-s'}(-\ell) - \mathcal{E}^*_{r}(\ell)\Big)b_{-s}(-\ell)\\&+ b^*_{-s}(-\ell)\Big(K(\ell)_{r,s'}b_{-s'}(-\ell) - \mathcal{E}_{r}(\ell)\Big)\\ &+ \Big(K(\ell)_{s,s'}b_{s'}(\ell) - \mathcal{E}_{-s}(-\ell)\Big)b^*_{r}(\ell) \Bigg)
    \end{aligned}\\
    = &-\sum\limits_{\ell \in \Z^3_*}\sum\limits_{r,s,s' \in L_{\ell}}A(\ell)_{r,s}\begin{aligned}[t]
        \Big(&K(\ell)_{s,s'}b_{r}(\ell)b^*_{s'}(\ell) - b_{r}(\ell) \mathcal{E}^*_{-s}(-\ell)\\ + &K(\ell)_{r,s'}b^*_{-s'}(-\ell)b_{-s}(-\ell) - \mathcal{E}^*_{r}(\ell)b_{-s}(-\ell)\\+& K(\ell)_{r,s'}b^*_{-s}(-\ell)b_{-s'}(-\ell) -b^*_{-s}(-\ell) \mathcal{E}_{r}(\ell)\\ + &K(\ell)_{s,s'}b_{s'}(\ell)b^*_{r}(\ell) - \mathcal{E}_{-s}(-\ell)b^*_{r}(\ell) \Big)
    \end{aligned}\\
     =&-\sum\limits_{\ell \in \Z^3_*}\sum\limits_{r,s,s' \in L_{\ell}}A(\ell)_{r,s}\begin{aligned}[t]
        \Big(&K(\ell)_{s,s'}b_{r}(\ell)b^*_{s'}(\ell)  + K(\ell)_{r,s'}b^*_{-s'}(-\ell)b_{-s}(-\ell) \\+& K(\ell)_{r,s'}b^*_{-s}(-\ell)b_{-s'}(-\ell) + K(\ell)_{s,s'}b_{s'}(\ell)b^*_{r}(\ell)\Big)
        \end{aligned}\nonumber\\
    &+\sum\limits_{\ell \in \Z^3_*}\sum\limits_{r,s \in L_{\ell}}A(\ell)_{r,s}\begin{aligned}[t]\Big(&b_{r}(\ell) \mathcal{E}^*_{-s}(-\ell) +\mathcal{E}^*_{r}(\ell)b_{-s}(-\ell) \\+&b^*_{-s}(-\ell) \mathcal{E}_{r}(\ell)+ \mathcal{E}_{-s}(-\ell)b^*_{r}(\ell) \Big)\end{aligned}.\label{eq:unrefcomerrq2k}
\end{alignat}
Here we represent the second sum (in (\ref{eq:unrefcomerrq2k})) as $E_{Q_2}(A(\ell))$, the commutation error, which can be further written as
\begin{equation}
    E_{Q_2}(A(\ell)) = \sum\limits_{\ell \in \Z^3_*}\sum\limits_{r,s \in L_{\ell}}A(\ell)_{r,s}\Big( \mathcal{E}^*_{r}(\ell)b_{-s}(-\ell) +b^*_{-s}(-\ell) \mathcal{E}_{r}(\ell)+b_{r}(\ell) \mathcal{E}^*_{-s}(-\ell)+ \mathcal{E}_{-s}(-\ell)b^*_{r}(\ell) \Big)\label{eq:Q2Kerr_no_mod}
\end{equation}
Then in the last two terms, we exchange the indices $r$ and $s$ and reflect all the momenta (i.e. $\ell\rightarrow -\ell,r\rightarrow -r,s\rightarrow -s$) to get
\begin{align}
    (\ref{eq:Q2Kerr_no_mod})&=\sum\limits_{\ell \in \Z^3_*}\sum\limits_{r,s \in L_{\ell}}A(\ell)_{r,s}\Big( \mathcal{E}^*_{r}(\ell)b_{-s}(-\ell) +b^*_{-s}(-\ell) \mathcal{E}_{r}(\ell)+b_{-s}(-\ell) \mathcal{E}^*_{r}(\ell)+ \mathcal{E}_{r}(\ell)b^*_{-s}(-\ell) \Big)\\
    &=\sum\limits_{\ell \in \Z^3_*}\sum\limits_{r,s \in L_{\ell}}A(\ell)_{r,s}\Big(\big\{\mathcal{E}^*_{r}(\ell), b_{-s}(-\ell)\big\} + \big\{\mathcal{E}_r(l), b^*_{-s}(-l)\big\}\Big).
\end{align}
Now we substitute this $E_{Q_2} $ in (\ref{eq:unrefcomerrq2k}) to have
\begin{align}
    (\ref{eq:unrefcomerrq2k})=&-\sum\limits_{\ell \in \Z^3_*}\sum\limits_{r,s,s' \in L_{\ell}}A(\ell)_{r,s}
        K(\ell)_{s,s'}\big(b_{r}(\ell)b^*_{s'}(\ell)+b_{s'}(\ell)b^*_{r}(\ell) \big)\nonumber\\ &-\sum\limits_{\ell \in \Z^3_*}\sum\limits_{r,s,s' \in L_{\ell}}A(\ell)_{r,s} K(\ell)_{r,s'}\big(b^*_{-s'}(-\ell)b_{-s}(-\ell) + b^*_{-s}(-\ell)b_{-s'}(-\ell)\big) + E_{Q_2}(A(\ell)).\label{eq:comwitherr}
\end{align}
Next we reflect all the momenta (i.e. $\ell\rightarrow -\ell,r\rightarrow -r,s\rightarrow -s,s'\rightarrow -s'$) in the second sum of (\ref{eq:comwitherr}) to have
\begin{align}
   -\sum\limits_{\ell \in \Z^3_*}\sum\limits_{r,s,s' \in L_{\ell}}A(\ell)_{r,s} K(\ell)_{r,s'}\big(b^*_{s'}(\ell)b_{s}(\ell) + b^*_{s}(\ell)b_{s'}(\ell)\big).
\end{align}
Then we do a sequence of identifications on the second term, first we exchange $s$ and $s'$ 
\begin{equation}
   -\sum\limits_{\ell \in \Z^3_*}\sum\limits_{r,s,s' \in L_{\ell}}A(\ell)_{r,s'} K(\ell)_{r,s}\big(b^*_{s}(\ell)b_{s'}(\ell) + b^*_{s'}(\ell)b_{s}(\ell)\big).
\end{equation}
Next, we exchange $s$ and $r$ to arrive at
\begin{equation}\label{eq:comQ2kalliden}
   -\sum\limits_{\ell \in \Z^3_*}\sum\limits_{r,s,s' \in L_{\ell}}A(\ell)_{s,s'} K(\ell)_{r,s}\big(b^*_{r}(\ell)b_{s'}(\ell) + b^*_{s'}(\ell)b_{r}(\ell)\big).
\end{equation}
Then substituting (\ref{eq:comQ2kalliden}) in (\ref{eq:comwitherr}) to arrive at
\begin{align}
    (\ref{eq:comwitherr})=&-\sum\limits_{\ell \in \Z^3_*}\sum\limits_{r,s,s' \in L_{\ell}}A(\ell)_{r,s}
        K(\ell)_{s,s'}b_{r}(\ell)b^*_{s'}(\ell)+\underbrace{A(\ell)_{r,s}
        K(\ell)_{s,s'}b_{s'}(\ell)b^*_{r}(\ell)}_{\text{(a)}}\nonumber\\ &-\sum\limits_{\ell \in \Z^3_*}\sum\limits_{r,s,s' \in L_{\ell}}A(\ell)_{s,s'} K(\ell)_{r,s}b^*_{r}(\ell)b_{s'}(\ell) + \underbrace{A(\ell)_{r,s}
        K(\ell)_{s,s'}b^*_{s'}(\ell)b_{r}(\ell)}_{\text{(b)}} + E_{Q_2}.\label{eq:Q2k_before_anticom}
\end{align}
And finally to interpret the terms as a matrix product, we exchange $r$ and $s'$ in terms (a) and (b) above to have 
\begin{align}
    (\ref{eq:Q2k_before_anticom}) = &-\sum\limits_{\ell \in \Z^3_*}\sum\limits_{r,s,s' \in L_{\ell}}\Big\{A(\ell)_
        ,K(\ell)\Big\}_{r,s}\big(b^*_{r}(\ell)b_{s}(\ell)+b_{r}(\ell)b^*_{s}(\ell) \big)\nonumber + E_{Q_2}(A(\ell))\\
        &= Q_1\left(\Big\{A(\ell)_
        ,K(\ell)\Big\}\right) + E_{Q_2}(A(\ell)).
\end{align}
\end{proof}
\newpage
\subsection{Transformation of quadratic operators}
We begin with $T^*_1Q_!(A(\ell))T_1$ and apply Duhamel's formula, 
\begin{align}
    T^*_1Q_1(A(\ell))T_1 &= Q_1(A(\ell)) + \bint\limits_0^1 \frac{d}{d\lambda}\left(T^*_{\lambda}Q_1(A(\ell))T_{\lambda}\right)d\lambda\label{eq:25}\\
    &=Q_1(A_k) + \bint\limits_0^1 T^*_{\lambda}[\mathcal{K},Q_1(A_k)]T_{\lambda}d\lambda.\label{eq:26}
\end{align}
Then from the lemma above, we get
\begin{align}
    \eqref{eq:26}&= Q_1(A_k) + \bint\limits_0^1 T^*_{\lambda}(Q_2(\{K_k,A_k\}) + E_{Q_1}(K_k,A_k))T_{\lambda}d\lambda\label{eq:27}\\
    &= Q_1(A_k) + \bint\limits_0^1 T^*_{\lambda}(Q_2(\{K_k,A_k\})T_{\lambda}d\lambda + \bint\limits_0^1 T^*_{\lambda}E_{Q_1}(K_k,A_k))T_{\lambda}d\lambda\label{eq:28}
\end{align}
\begin{lemma}[Action of $T_\lambda$ on quadratic operators]\label{lem:4}
    
    \begin{align}
        T^*_{\lambda}Q_1(A_K)T_{\lambda} 
        &=Q_1(A_k) + \bint\limits_0^{\lambda} T^*_{\lambda'}(Q_2(\{K_k,A_k\})T_{\lambda'}d\lambda' + \bint\limits_0^{\lambda} T^*_{\lambda'}E_{Q_1}(K_k,A_k))T_{\lambda'}d\lambda'\label{eq:29}\\
        T^*_{\lambda}Q_2(A_K)T_{\lambda} 
        &= Q_2(A_k) + \bint\limits_0^{\lambda} T^*_{\lambda'}(Q_1(\{K_k,A_k\})T_{\lambda'}d\lambda' + \bint\limits_0^{\lambda} T^*_{\lambda'}E_{Q_2}(K_k,A_k))T_{\lambda'}d\lambda'\\
        &+ 
        \bint\limits_0^{\lambda} T^*_{\lambda'}\epsilon(\ell)T_{\lambda'}d\lambda'\label{eq:30}
    \end{align}
\end{lemma}
\begin{proof}
    We prove the above by using the Duhamel's formula, redoing the above computation \eqref{eq:25} through \eqref{eq:28}.
\end{proof}
\begin{proposition}[Action of $T_1$ on $Q_2(A_k)$]

\begin{align}
    T^*_1Q_2(A_K)T_1 &= Q_2\left(\sum\limits_{m\geq0}\frac{\Theta^{2m}_K(A_k)}{(2m)!}\right) + Q_1\left(\sum\limits_{m\geq0}\frac{\Theta^{2m+1}_K(A_k)}{(2m+1)!}\right) \nonumber\\
        &+\bint\limits_0^1\bint\limits_0^\lambda\ldots \bint\limits_0^{\lambda_{n-1}} T^*_{\lambda_n}(Q_{\sigma(n)}(\Theta^n_K(A_k))T_{\lambda_n}d\lambda_n\ldots d\lambda_1 d\lambda\\ &+ 
        \sum\limits_{n\geq0}\bint\limits_0^{\lambda}\bint\limits_0^{\lambda_1}\cdots\bint\limits_0^{\lambda_2n+1} T^*_{\lambda}\Theta^n_K(\epsilon(\ell))T_{\lambda}d\lambda d\lambda_1\ldots d\lambda_{2n+1} +\sum\limits_{i=1}^{n}E_i
    \end{align}
    where $\sigma(n) = \begin{cases}
        1 &\text{for n even}\\
        2 &\text{for n odd}.
    \end{cases}$. 
\end{proposition}
\begin{proof}
    We have, from \eqref{eq:28},
    \begin{equation}
        T^*_1Q_1(A_K)T_1 = Q_1(A_k) + \bint\limits_0^1 T^*_{\lambda}(Q_2(\{K_k,A_k\})T_{\lambda}d\lambda + \bint\limits_0^1 T^*_{\lambda}E_{Q_1}(K_k,A_k))T_{\lambda}d\lambda\label{eq:32}
    \end{equation}
    We use \eqref{eq:30} from Lemma \ref{lem:4} above to arrive at
    \begin{align}
        \eqref{eq:32} &= Q_1(A_k) + \frac{1}{1!}Q_2(\{K_k,A_k\}) +\bint\limits_0^1\bint\limits_0^\lambda T^*_{\lambda_1}(Q_1(\{K_k,\{K_k,A_k\}\})T_{\lambda_1}d\lambda_1 d\lambda \nonumber\\&+ \bint\limits_0^1\bint\limits_0^\lambda T^*_{\lambda_1}(E_{Q_2}(K_k,\{K_k,A_k\})T_{\lambda_1}d\lambda_1 d\lambda +\bint\limits_0^1 T^*_{\lambda}E_{Q_1}(K_k,A_k))T_{\lambda}d\lambda\label{eq:33}.
    \end{align}
    Next we use \eqref{eq:29} from Lemma \ref{lem:4}
    \begin{align}
        \eqref{eq:33} &= Q_1(A_k) + \frac{1}{1!}Q_2(\{K_k,A_k\})+ \frac{1}{2!}Q_1(\{K_k,\{K_k,A_k\}\}) \nonumber\\
        &+\bint\limits_0^1\bint\limits_0^\lambda \bint\limits_0^{\lambda_1} T^*_{\lambda_2}(Q_2(\{K_k,\{K_k,\{K_k,A_k\}\}\})T_{\lambda_2}d\lambda_2d\lambda_1 d\lambda \nonumber\\
        &+\bint\limits_0^1\bint\limits_0^\lambda \bint\limits_0^{\lambda_1} T^*_{\lambda_2}E_{Q_1}(K_k,\{K_k,\{K_k,A_k\}\})T_{\lambda_2}d\lambda_2d\lambda_1 d\lambda \nonumber\\
        &+ \bint\limits_0^1\bint\limits_0^\lambda T^*_{\lambda_1}E_{Q_2}(K_k,\{K_k,A_k\})T_{\lambda_1}d\lambda_1 d\lambda +\bint\limits_0^1 T^*_{\lambda}E_{Q_1}(K_k,A_k)T_{\lambda}d\lambda\label{eq:34}.
    \end{align}
    For our convenience, we introduce the following notation for writing the nested anti-commutators 
    \begin{equation}
        \Theta^n_K(A(\ell)) = \underbrace{\{K(\ell),\{\ldots,\{K(\ell)}_\textrm{n times},A(\ell)\}\ldots\}\}\\
    \end{equation}
    with
    \begin{equation}
        \Theta^0_K(A_k) = A_k.
    \end{equation}
    We also introduce another notation for the error terms 
    \begin{equation}
    E_n(K(\ell), A(\ell)) =     
        \begin{cases}
        \bint\limits_{\Delta^{2n}}T^*_{\lambda}E_{Q_1}\left(\Theta^{2n}_{K}(A(\ell)),K(\ell)\right)T_{\lambda}d\lambda & \text{for n even}\\
        \bint\limits_{\Delta^{2n-1}}T^*_{\lambda}E_{Q_2}\left(\Theta^{2n-1}_{K}(A(\ell)),K(\ell)\right)T_{\lambda}d\lambda & \text{for n odd}
        \end{cases}.
    \end{equation}
    Then after multiple interations we arrive at
    \begin{align}
        T^*_1Q_1(A_K)T_1 &=  Q_1(\Theta^0_K(A_k)) + \frac{1}{1!}Q_2(\Theta^1_K(A_k))+ \frac{1}{2!}Q_1(\Theta^2_K(A_k)) \nonumber\\
        &+ \frac{1}{3!}Q_1(\Theta^3_K(A_k)) + \ldots \nonumber \\
        &+\bint\limits_0^1\bint\limits_0^\lambda\ldots \bint\limits_0^{\lambda_{n-1}} T^*_{\lambda_n}(Q_{\sigma(n)}(\Theta^n_K(A_k))T_{\lambda_n}d\lambda_n\ldots d\lambda_1 d\lambda \nonumber\\
        &+ E_n(K_k, A_k) + E_{n-1}(K_k, A_k)+\dots+E_1(K_k, A_k)\\
        &= Q_1\left(\sum\limits_{m\geq0}\frac{\Theta^{2m}_K(A_k)}{(2m)!}\right) + Q_2\left(\sum\limits_{m\geq0}\frac{\Theta^{2m+1}_K(A_k)}{(2m+1)!}\right) \nonumber\\
        &+\bint\limits_0^1\bint\limits_0^\lambda\ldots \bint\limits_0^{\lambda_{n-1}} T^*_{\lambda_n}(Q_{\sigma(n)}(\Theta^n_K(A_k))T_{\lambda_n}d\lambda_n\ldots d\lambda_1 d\lambda + \sum\limits_{i=1}^{n}E_i(K_k, A_k)
    \end{align}
    where $\sigma(n) = \begin{cases}
        1 &\text{for n even}\\
        2 &\text{for n odd}.
    \end{cases}$.
\end{proof}
\begin{proposition}[Final Expansion]
\textcolor{red}{Change $A_K$ by $\Delta$}
\begin{align}
    \eva{\Omega, T_1^*\left(n_q+n_{-q}\right)T_1\Omega} &= \Bigg<\Omega,\bigg(Q_2\left(\sum\limits_{m\geq0}\frac{\Theta^{2m}_K(A_k)}{(2m)!}\right) + Q_1\left(\sum\limits_{m\geq0}\frac{\Theta^{2m+1}_K(A_k)}{(2m+1)!}\right) \nonumber\\
        &+\left.\bint\limits_0^1\bint\limits_0^\lambda\ldots \bint\limits_0^{\lambda_{n-1}} T^*_{\lambda_n}(Q_{\sigma(n)}(\Theta^n_K(A_k))T_{\lambda_n}d\lambda_n\ldots d\lambda_1 d\lambda\right.\\ &+ \sum\limits_{n\geq0}\bint\limits_0^{\lambda}\bint\limits_0^{\lambda_1}\cdots\bint\limits_0^{\lambda_2n+1} T^*_{\lambda}\Theta^n_K(\epsilon(\ell))T_{\lambda}d\lambda d\lambda_1\ldots d\lambda_{2n+1} +\sum\limits_{i=1}^{n}E_i\bigg)\Omega\Bigg>
\end{align}
    
\end{proposition}
\begin{proof}
    \textcolor{red}{to be filled in}
\end{proof}
%\section{This is an example for first level head---section head}\label{sec3}
%\subsection{This is an example for second level head---subsection head}\label{subsec2}
%\subsubsection{This is an example for third level head---subsubsection head}\label{subsubsec2}
%\section{Equations}\label{sec4}
%Notice the use of \verb+\nonumber+ in the align environment at the end of each line, except the last, so as not to produce equation numbers on lines where no equation numbers are required. The \verb+\label{}+ command should only be used at the last line of an align environment where \verb+\nonumber+ is not used.

%\begin{theorem}[Theorem subhead]\label{thm1}
%\end{theorem}

%\begin{proposition}
%\end{proposition}

%\begin{example}
%\end{example}

%\begin{remark}
%\end{remark}


%\begin{definition}[Definition sub head]
%\end{definition}

%\begin{proof}

%\end{proof}

%\begin{proof}[Proof of Theorem~{\upshape\ref{thm1}}]
%\end{proof}
\section{Error bounds}
\begin{definition}
    \begin{align}
        \Xi_\lambda(k) &\coloneq \expval{T_\lambda\Omega, a^*_ka_kT_\lambda\Omega}\\
        \Xi_\lambda &\coloneq \sup\limits_{k}\expval{T_\lambda\Omega, a^*_ka_kT_\lambda\Omega}
    \end{align}
\end{definition}
\begin{lemma}
    \begin{equation}
    \expval{T_\lambda\Omega, E_n(K_k,A_k)T_\lambda\Omega}\leq e^{||k||} \Xi_\lambda  
    \end{equation}
    
\end{lemma}

%\begin{appendices}

%\section{Section title of first appendix}\label{secA1}


%\end{appendices}


\bibliography{sn-bibliography}% common bib file
%% if required, the content of .bbl file can be included here once bbl is generated
%%\input sn-article.bbl


\end{document}
