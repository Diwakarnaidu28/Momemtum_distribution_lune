%%%%%%%%%%%%%%%%%%%%%%%%%%%%%%%%%%%%%%%%%%%%%%%%%%%%%%%%%%%%%%%%%%%%%%
%%                                                                 %%
%% Please do not use \input{...} to include other tex files.       %%
%% Submit your LaTeX manuscript as one .tex document.              %%
%%                                                                 %%
%% All additional figures and files should be attached             %%
%% separately and not embedded in the \TeX\ document itself.       %%
%%                                                                 %%
%%%%%%%%%%%%%%%%%%%%%%%%%%%%%%%%%%%%%%%%%%%%%%%%%%%%%%%%%%%%%%%%%%%%%

%%\documentclass[referee,sn-basic]{sn-jnl}% referee option is meant for double line spacing
%%=======================================================%%
%% to print line numbers in the margin use lineno option %%
%%=======================================================%%

%%\documentclass[lineno,sn-basic]{sn-jnl}% Basic Springer Nature Reference Style/Chemistry Reference Style

%%======================================================%%
%% to compile with pdflatex/xelatex use pdflatex option %%
%%======================================================%%

%%\documentclass[pdflatex,sn-basic]{sn-jnl}% Basic Springer Nature Reference Style/Chemistry Reference Style


%%Note: the following reference styles support Namedate and Numbered referencing. By default the style follows the most common style. To switch between the options you can add or remove “Numbered” in the optional parenthesis. 
%%The option is available for: sn-basic.bst, sn-vancouver.bst, sn-chicago.bst, sn-mathphys.bst. %  
 
%%\documentclass[sn-nature]{sn-jnl}% Style for submissions to Nature Portfolio journals
%%\documentclass[sn-basic]{sn-jnl}% Basic Springer Nature Reference Style/Chemistry Reference Style
\documentclass[sn-mathphys,Numbered]{sn-jnl}% Math and Physical Sciences Reference Style
%%\documentclass[sn-aps]{sn-jnl}% American Physical Society (APS) Reference Style
%%\documentclass[sn-vancouver,Numbered]{sn-jnl}% Vancouver Reference Style
%%\documentclass[sn-apa]{sn-jnl}% APA Reference Style 
%%\documentclass[sn-chicago]{sn-jnl}% Chicago-based Humanities Reference Style
%%\documentclass[default]{sn-jnl}% Default
%%\documentclass[default,iicol]{sn-jnl}% Default with double column layout

%%%% Standard Packages
%%<additional latex packages if required can be included here>
\usepackage{amsmath, amssymb, amsfonts, physics, braket, hhline, dsfont, mathtools, cancel, bigints}
\usepackage{pgfplots, subcaption, floatrow, footnote, adjustbox,float,fancyvrb}
\usepackage{graphicx, grffile, epsfig, listings,romannum, hyperref}
\usepackage{verbatim}
\usepackage{textcomp}
\usepackage{pdfpages}
\usepackage{accents}
\usepackage{tikz-cd}
\usepackage{multirow}%
\usepackage{amsthm}%
\usepackage{mathrsfs}%
\usepackage[title]{appendix}%
\usepackage{xcolor}%
\usepackage{textcomp}%
\usepackage{manyfoot}%
\usepackage{booktabs}%
\usepackage{algorithm}%
\usepackage{algorithmicx}%
\usepackage{algpseudocode}%
\pgfplotsset{compat=1.9}
\usetikzlibrary{shapes, arrows.meta, positioning, shapes.geometric}
\usepackage[capitalise]{cleveref}
%%%%
%%%%%%%%%%%%%%%%%%%%%%%%%%%%%%%%%
\DeclareMathOperator{\R}{\mathbb{R}}
\DeclareMathOperator{\C}{\mathbb{C}}
\DeclareMathOperator{\N}{\mathbb{N}}

\DeclareMathOperator{\QQ}{\mathcal{Q}}
\DeclareMathOperator{\HH}{\mathcal{H}}
\DeclareMathOperator{\LL}{\mathcal{L}}
\DeclareMathOperator{\KK}{\mathcal{K}}

\DeclareMathOperator{\SH}{\mathscr{H}}
\DeclareMathOperator{\Psis}{\Psi^*}
\newcommand{\bint}{\bigintssss}
\newcommand\Item[1][]{%
  \ifx\relax#1\relax  \item \else \item[#1] \fi
  \abovedisplayskip=0pt\abovedisplayshortskip=0pt~\vspace*{-\baselineskip}}
%%%%%%%%%%%%%%%%%%%%%%%%%%%%%%%%%%%%%%%%%%%%%%%%%%%
% THEOREMSTYLES
\theoremstyle{plain}
\newtheorem{theorem}{Theorem}[section]
\newtheorem{lemma}[theorem]{Lemma}
\newtheorem{corollary}[theorem]{Corollary}
\newtheorem{observation}[theorem]{Observation}
\newtheorem{proposition}[theorem]{Proposition}

\theoremstyle{definition}
\newtheorem{definition}[theorem]{Definition}
\newtheorem{problem}[theorem]{Problem}
\newtheorem{assumption}[theorem]{Assumption}
\newtheorem{example}[theorem]{Example}

\theoremstyle{remark}
\newtheorem{claim}[theorem]{Claim}
\newtheorem{remark}[theorem]{Remark}

% UNNUMBERED VERSIONS
\theoremstyle{plain}
\newtheorem*{theorem*}{Theorem}
\newtheorem*{lemma*}{Lemma}
\newtheorem*{corollary*}{Corollary}
\newtheorem*{proposition*}{Proposition}


\theoremstyle{definition}
\newtheorem*{definition*}{Definition}
\newtheorem*{problem*}{Problem}
\newtheorem*{assumption*}{Assumption}
\newtheorem*{example*}{Example}

\theoremstyle{remark}
\newtheorem*{claim*}{Claim}
\newtheorem*{remark*}{Remark}
%%\newtheorem{theorem}{Theorem}[section]% meant for sectionwise numbers
%% optional argument [theorem] produces theorem numbering sequence instead of independent numbers for Proposition
%%%%%%%%%%%%%%%%%%%%%%%%%%%%%%%%%%%%%%

\begin{document}

\title[Article Title]{Occupation Density}

%%=============================================================%%
%% Prefix	-> \pfx{Dr}
%% GivenName	-> \fnm{Joergen W.}
%% Particle	-> \spfx{van der} -> surname prefix
%% FamilyName	-> \sur{Ploeg}
%% Suffix	-> \sfx{IV}
%% NatureName	-> \tanm{Poet Laureate} -> Title after name
%% Degrees	-> \dgr{MSc, PhD}
%% \author*[1,2]{\pfx{Dr} \fnm{Joergen W.} \spfx{van der} \sur{Ploeg} \sfx{IV} \tanm{Poet Laureate} 
%%                 \dgr{MSc, PhD}}\email{iauthor@gmail.com}
%%=============================================================%%

%\author[1]{\fnm{Niels} \sur{Benedikter}}\email{niels.benedikter@unimi.it}

%\author[1]{\fnm{Diwakar} \sur{Naidu}}\email{diwakar.naidu@unimi.it}
%\equalcont{These authors contributed equally to this work.}

%\author[1,2]{\fnm{Third} \sur{Author}}\email{iiiauthor@gmail.com}
%\equalcont{These authors contributed equally to this work.}

%\affil[1]{\orgdiv{Department}, \orgname{Organization}, \orgaddress{\street{Street}, \city{City}, \postcode{100190}, \state{State}, \country{Country}}}

%\affil[2]{\orgdiv{Department}, \orgname{Organization}, \orgaddress{\street{Street}, \city{City}, \postcode{10587}, \state{State}, \country{Country}}}

%\affil[3]{\orgdiv{Department}, \orgname{Organization}, \orgaddress{\street{Street}, \city{City}, \postcode{610101}, \state{State}, \country{Country}}}

%%==================================%%
%% sample for unstructured abstract %%
%%==================================%%

\abstract{The }

%%================================%%
%% Sample for structured abstract %%
%%================================%%

% \abstract{\textbf{Purpose:} The abstract serves both as a general introduction to the topic and as a brief, non-technical summary of the main results and their implications. The abstract must not include subheadings (unless expressly permitted in the journal's Instructions to Authors), equations or citations. As a guide the abstract should not exceed 200 words. Most journals do not set a hard limit however authors are advised to check the author instructions for the journal they are submitting to.
% 
% \textbf{Methods:} The abstract serves both as a general introduction to the topic and as a brief, non-technical summary of the main results and their implications. The abstract must not include subheadings (unless expressly permitted in the journal's Instructions to Authors), equations or citations. As a guide the abstract should not exceed 200 words. Most journals do not set a hard limit however authors are advised to check the author instructions for the journal they are submitting to.
% 
% \textbf{Results:} The abstract serves both as a general introduction to the topic and as a brief, non-technical summary of the main results and their implications. The abstract must not include subheadings (unless expressly permitted in the journal's Instructions to Authors), equations or citations. As a guide the abstract should not exceed 200 words. Most journals do not set a hard limit however authors are advised to check the author instructions for the journal they are submitting to.
% 
% \textbf{Conclusion:} The abstract serves both as a general introduction to the topic and as a brief, non-technical summary of the main results and their implications. The abstract must not include subheadings (unless expressly permitted in the journal's Instructions to Authors), equations or citations. As a guide the abstract should not exceed 200 words. Most journals do not set a hard limit however authors are advised to check the author instructions for the journal they are submitting to.}

\keywords{keyword1, Keyword2, Keyword3, Keyword4}

%%\pacs[JEL Classification]{D8, H51}

%%\pacs[MSC Classification]{35A01, 65L10, 65L12, 65L20, 65L70}

\maketitle

\section{Introduction}\label{sec1}
 

\section{Computations}\label{sec2}
Consider a trial state $\Psi_{trial}$ such that $\braket{\Psi_{trial},H\Psi_{trial}} = E_{HF} + E_{RPA}+ o(\hbar) $, where $E_{HF}$ is the Hartree-Fock energy and $E_{RPA}$ is the correlation energy from RPA.

We need to calculate $\braket{\Psi_{trial},a^*_la_l\Psi_{trial}},\, l\in\mathbb{Z}^3$. Here the trial state $\Psi_{trial}= Re^k\Omega$, where 
\begin{equation}
    \Omega = \frac{1}{\sqrt{N!}}\text{det}\left(\frac{1}{(2\pi)^{3/2}}e^{ik_j\cdot x_i}\right)^N_{j,i=1}.
\end{equation}
Then we have
\begin{align}
    \braket{\Psi_{trial},a^*_la_l\Psi_{trial}} &= \braket{\Psi_{trial},a^*_la_l\Psi_{trial}} 
\end{align}
We define the pair operators as
\begin{align}
    b_p(k) &= a_{p-k}a_{p}\\
    b^*_p(k) &= a^*_{p}a^*_{p-k}
\end{align}
\begin{lemma}[Quasi-Bosonic commutation relation]
   \begin{align}
       [b_{p}(k),b_{q}(l)] &= [b^*_{p}(k),b^*_{q}(l)] = 0\\
       [b_{p}(k),b^*_{q}(l)] &= \delta_{p,q}\delta(k,l) - \epsilon_{p,q}(k,l),
   \end{align} where $\epsilon_{p,q}(k,l) = \left(\delta_{p,q}a^*_{q-l}a_{p-k} + \delta_{p-k,q-l}a^*_{q}a_{p}\right)$ 
    \end{lemma}
\begin{proof}
    \begin{align}
        [b_{p}(k),b^*_{q}(l)] &= [a_{p-k}a_{p},a^*_{q}a^*_{q-l}]\\
        &= a_{p-k}[a_p,a^*_{q}a^*_{q-l}] + [a_{p-k},a^*_{q}a^*_{q-l}]a_{p}\\
        &= a_{p-k}\left\{a_p,a^*_{q}\right\}a^*_{q-l} - a_{p-k}a^*_{q}\{a_{p},a^*_{q-l}\}\nonumber \\ &+ \{a_{p-k},a^*_q\}a^*_{q-l}a_{p} - a^*_{q}\{a_{p-k},a^*_{q-l}\}a_{p}\\
        &=\delta_{p,q}a_{p-k}a^*_{q-l} + \delta_{p-k,q-l}a^*_{q}a_{p}\\
        &= \delta_{p,q}\delta(k,l)-\left(\delta_{p,q}a^*_{q-l}a_{p-k} + \delta_{p-k,q-l}a^*_{q}a_{p}\right)
    \end{align}
Here, we denote the error term as $\epsilon_{p,q}(k,l) = \left(\delta_{p,q}a^*_{q-l}a_{p-k} + \delta_{p-k,q-l}a^*_{q}a_{p}\right)$ 
\end{proof}
We have $T_1 = e^{\mathcal{K}}$, where 
\begin{equation}
\mathcal{K} = \frac{1}{2}\sum\limits_{l\in \mathbb{Z}^3_*}\sum\limits_{r,s\in L_l}K(l)_{r,s}\left(b_r(l)b_{-s}(-l)-b^*_{-s}(-l)b^*_{r}(l)\right)
\end{equation}
\begin{lemma}[Commutator between $\mathcal{K} $ and pair operators]
\begin{align}
    [\mathcal{K}, b_p(k)] &=\sum\limits_{r\in L_{k}}K(k)_{r,p}b^*_{r}(k) + \mathcal{E}^*_{p}(k)\label{eq:13} \\
    [\mathcal{K}, b^*_p(k)] &=\sum\limits_{r\in L_{k}}K(k)_{p,r}b_{r}(k) + \mathcal{E}_{p}(k)\label{eq:14},
\end{align}
    where, 
\begin{align}
    \mathcal{E}^*_{p}(k) &= \frac{1}{2}\sum\limits_{l\in \mathbb{Z}^3_*}\sum\limits_{r,s\in L_l}K(l)_{r,s}\left\{\epsilon_{p,s}(k,l),b^*_{-r}(-l)\right\}\\
    \mathcal{E}_{p}(k) &= \frac{1}{2}\sum\limits_{l\in \mathbb{Z}^3_*}\sum\limits_{r,s\in L_l}K(l)_{r,s}\left\{\epsilon_{p,s}(k,l),b_{-r}(-l)\right\} 
\end{align}
\end{lemma}
\begin{proof}
    We start with
    \begin{align}
        [\mathcal{K}, b_p(k)] &=\left[\frac{1}{2}\sum\limits_{l\in \mathbb{Z}^3_*}\sum\limits_{r,s\in L_l}K(l)_{r,s}\left(b_r(l)b_{-s}(-l)-b^*_{-s}(-l)b^*_{r}(l)\right),b_p(k)\right]\nonumber\\
        &=\frac{1}{2}\sum\limits_{l\in \mathbb{Z}^3_*}\sum\limits_{r,s\in L_l}K(l)_{r,s}\left[\left(b_r(l)b_{-s}(-l)-b^*_{-s}(-l)b^*_{r}(l)\right),b_p(k)\right]\nonumber\\
        &=\frac{1}{2}\sum\limits_{l\in \mathbb{Z}^3_*}\sum\limits_{r,s\in L_l}K(l)_{r,s}\left[\left(-b^*_{-s}(-l)b^*_{r}(l)\right),b_p(k)\right]\nonumber\\
        &=\frac{1}{2}\sum\limits_{l\in \mathbb{Z}^3_*}\sum\limits_{r,s\in L_l}K(l)_{r,s}\left(b^*_{-s}(-l)\left[b_p(k),b^*_{r}(l)\right] + \left[b_p(k),b^*_{-s}(-l)\right]b^*_{r}(l) \right)
    \end{align}
    Substituting $l\rightarrow -l, r\rightarrow -r $ and $ s\rightarrow -s$ in the first term above, we get
    \begin{align}
        &\phantom{=}\frac{1}{2}\sum\limits_{l\in \mathbb{Z}^3_*}\sum\limits_{r,s\in L_l}K(-l)_{-r,-s}\left(b^*_{s}(l)\left[b_p(k),b^*_{-r}(-l)\right]\right) \nonumber\\
        &+\frac{1}{2}\sum\limits_{l\in \mathbb{Z}^3_*}\sum\limits_{r,s\in L_l}K(l)_{r,s}\left( \left[b_p(k),b^*_{-s}(-l)\right]b^*_{r}(l) \right)\nonumber\\
        &=\frac{1}{2}\sum\limits_{l\in \mathbb{Z}^3_*}\sum\limits_{r,s\in L_l}K(l)_{r,s}\left(b^*_{s}(l)\left[b_p(k),b^*_{-r}(-l)\right] + \left[b_p(k),b^*_{-s}(-l)\right]b^*_{r}(l) \right)\nonumber\\
        &= \frac{1}{2}\sum\limits_{l\in \mathbb{Z}^3_*}\sum\limits_{r,s\in L_l}K(l)_{r,s}\left\{\left[b_p(k),b^*_{-s}(-l)\right],b^*_{r}(l) \right\}\nonumber\\
        &= \frac{1}{2}\sum\limits_{l\in \mathbb{Z}^3_*}\sum\limits_{r,s\in L_l}K(l)_{r,s}\left\{(\delta_{p,-s}\delta(k,-l) + \epsilon_{p,-s}(k,-l)),b^*_{r}(l) \right\}\nonumber\\
        &= \frac{1}{2}\sum\limits_{l\in \mathbb{Z}^3_*}\sum\limits_{r,s\in L_l}K(l)_{r,s}\left\{(\delta_{p,-s}\delta(k,-l),b^*_{r}(l) \right\} + \frac{1}{2}\sum\limits_{l\in \mathbb{Z}^3_*}\sum\limits_{r,s\in L_l}K(l)_{r,s}\left\{\epsilon_{p,-s}(k,-l)),b^*_{r}(l) \right\}\nonumber\\
        &=\sum\limits_{r\in L_{-k}}K(-k)_{r,-p}b^*_r(-k) + \frac{1}{2}\sum\limits_{l\in \mathbb{Z}^3_*}\sum\limits_{r,s\in L_l}K(l)_{r,s}\left\{\epsilon_{p,-s}(k,-l)),b^*_{r}(l) \right\}\nonumber
    \end{align}
    Similarly we substitute $l\rightarrow -l, r\rightarrow -r $ and $ s\rightarrow -s$ in the second term above. 
    \begin{equation}
        \sum\limits_{r\in L_{-k}}K(-k)_{r,-p}b^*_r(-k) + \frac{1}{2}\sum\limits_{l\in \mathbb{Z}^3_*}\sum\limits_{r,s\in L_l}K(l)_{r,s}\left\{\epsilon_{p,s}(k,l)),b^*_{-r}(-l) \right\}
    \end{equation}
    We denote the second (error) term above by \begin{equation}
        \mathcal{E}^*_p(k)= \frac{1}{2}\sum\limits_{l\in \mathbb{Z}^3_*}\sum\limits_{r,s\in L_l}K(l)_{r,s}\left\{\epsilon_{p,s}(k,l)),b^*_{-r}(-l) \right\}
    \end{equation}
    and we get
    \begin{align}
        &\phantom{=}\sum\limits_{r\in L_{-k}}K(-k)_{r,-p}b^*_r(-k) + \mathcal{E}^*_p(k)\nonumber\\
        &=\sum\limits_{-r\in L_{k}}K(-k)_{-r,-p}b^*_{-r}(-k) + \mathcal{E}^*_p(k)\nonumber
    \end{align}
    Again substituting, $-k\rightarrow k, -r\rightarrow r $ and $ -p\rightarrow p$ in the first term above, we get
    \begin{equation}
        \sum\limits_{r\in L_{k}}K(k)_{r,p}b^*_{r}(k) + \mathcal{E}^*_p(k)
    \end{equation}
    Similarly, one can prove the other statement for $[\mathcal{K}, b^*_p(k)]$.
\end{proof}
Next we define the quadratic operators as 
\begin{align}
    Q_1(A_k)&=\sum\limits_{p,q \in L_{k}}A(k)_{p,q} \left(b^*_p(k)b_{q}(k)+b_{p}(k)b^*_{q}(k)\right)\\
    Q_2(B_k)&=  \sum\limits_{p,q \in L_{k}}B(k)_{p,q} \left(b^*_p(k)b^*_{q}(k)+b_{p}(k)b_{q}(k)\right)
\end{align}
\begin{lemma}[Commutator between $\mathcal{K} $ and the quadratic operators]
\begin{align}
    [\mathcal{K}, Q_1(A_k)] &= Q_2(\{K_k,A_k\}) + E_{Q_1}(A_k)\\
    [\mathcal{K}, Q_2(B_k)] &= Q_1(\{K_k,B_k\}) + E_{Q_2}(B_k)
\end{align}
 where,
 \begin{align}
     E_{Q_1}(A_k)&= \sum\limits_{p,q\in L_k}A(k)_{p,q}\left(\mathcal{E}_p(k)b_q(k)+b_p(k)\mathcal{E}_q(k) +b^*_p(k)\mathcal{E}^*_q(k) +\mathcal{E}^*_p(k)b^*_q(k)\right)\\
     E_{Q_2}(B_k)&= \sum\limits_{p,q\in L_k}B(k)_{p,q}\left(\mathcal{E}_p(k)b^*_q(k)+b^*_p(k)\mathcal{E}_q(k) +b_p(k)\mathcal{E}^*_q(k) +\mathcal{E}^*_p(k)b_q(k)\right)
 \end{align}
\end{lemma}
\begin{proof}
We begin with
\begin{align}
    [\mathcal{K}, Q_2(B_k)]&=  \left[\mathcal{K}, \sum\limits_{p,q \in L_{k}}B(k)_{p,q} \left(b^*_p(k)b^*_{q}(k)+b_{p}(k)b_{q}(k)\right)\right]\nonumber\\
    &=\sum\limits_{p,q \in L_{k}}B(k)_{p,q}\left[\mathcal{K},  \left(b^*_p(k)b^*_{q}(k)+b_{p}(k)b_{q}(k)\right)\right]\nonumber\\
    &=\sum\limits_{p,q \in L_{k}}B(k)_{p,q}\left([\mathcal{K}, b^*_p(k)b^*_{q}(k)]+[\mathcal{K},b_{p}(k)b_{q}(k)]\right)\nonumber\\
    &=\sum\limits_{p,q \in L_{k}}B(k)_{p,q}\left([\mathcal{K}, b^*_p(k)]b^*_{q}(k) + b^*_p(k)[\mathcal{K}, b^*_{q}(k)]\right.\nonumber\\
    &+\left.[\mathcal{K},b_{p}(k)]b_{q}(k) + b_{p}(k)[\mathcal{K},b_{q}(k)]\right)\nonumber\\
\end{align}
Now we use the commutation relation from \eqref{eq:13} and \eqref{eq:14} to get
\begin{align}
    &\phantom{=}\sum\limits_{p,q \in L_{k}}B(k)_{p,q}\left(\left(\sum\limits_{r\in L_{k}}K(k)_{p,r}b_{r}(k) + \mathcal{E}_{p}(k)\right)b^*_{q}(k) + b^*_p(k)\left(\sum\limits_{r\in L_{k}}K(k)_{q,r}b_{r}(k) + \mathcal{E}_{q}(k)\right)\right.\nonumber\\
    &+\left.\left(\sum\limits_{r\in L_{k}}K(k)_{r,p}b^*_{r}(k) + \mathcal{E}^*_{p}(k)\right)b_{q}(k) + b_{p}(k)\left(\sum\limits_{r\in L_{k}}K(k)_{r,q}b^*_{r}(k) + \mathcal{E}^*_{q}(k)\right)\right)\nonumber\\
    &=\sum\limits_{p,q,r \in L_{k}}B(k)_{p,q}\left(K(k)_{p,r}b_{r}(k)b^*_{q}(k) + b^*_p(k)K(k)_{q,r}b_{r}(k) + 
    K(k)_{r,p}b^*_{r}(k) b_{q}(k) + b_{p}(k)K(k)_{r,q}b^*_{r}(k) \right)\nonumber\\
    &+\sum\limits_{p,q\in L_k}B(k)_{p,q}\left(\mathcal{E}_p(k)b^*_q(k)+b^*_p(k)\mathcal{E}_q(k) +b_p(k)\mathcal{E}^*_q(k) +\mathcal{E}^*_p(k)b_q(k)\right).
\end{align}
Nest, we denote the second sum term as the error: $E_{Q_2}(B_k)$ and we arrive at
\begin{align}
    \sum\limits_{p,q,r \in L_{k}}B(k)_{p,q}\left(K(k)_{p,r}b_{r}(k)b^*_{q}(k) \right.&+\left. b^*_p(k)K(k)_{q,r}b_{r}(k)\right. \nonumber\\
    +\left. K(k)_{r,p}b^*_{r}(k) b_{q}(k)\right. &+\left. b_{p}(k)K(k)_{r,q}b^*_{r}(k) \right) + E_{Q_2}(B_k).\label{eq:1000} 
\end{align}
Next we remove the duplicate index and write it in a condensed way
\begin{align}
    \sum\limits_{p,q,r \in L_{k}}B(k)_{p,q}\left(K(k)_{p,r}b_{r}(k)b^*_{q}(k) \right.&+\left. b^*_p(k)K(k)_{q,r}b_{r}(k)\right. \nonumber\\
    +\left. K(k)_{r,p}b^*_{r}(k) b_{q}(k)\right. &+\left. b_{p}(k)K(k)_{r,q}b^*_{r}(k) \right) + E_{Q_2}(B_k)\\
    = \sum\limits_{p,q \in L_{k}} \left(\{K_k,B_k\}_{p,q}b_{p}(k)b^*_{q}(k) \right.&+\left. \{K_k,B_k\}_{p,q}b^*_{p}(k) b_{q}(k)\right) + E_{Q_2}(B_k)\\
    = \sum\limits_{p,q \in L_{k}}\{K_k,B_k\}_{p,q} \left(b_{p}(k)b^*_{q}(k) \right.&+\left.b^*_{p}(k) b_{q}(k)\right) + E_{Q_2}(B_k)\\
    = Q_1(\{K_k,B_k\}) + E_{Q_2}(B_k).
\end{align}
Similarly one can prove $[\mathcal{K}, Q_1(A_k)]$.
\end{proof}
\subsection{Transformation of quadratic operators}
We begin with $T^*_1Q_!(A_K)T_1$ and apply Duhamel's formula, 
\begin{align}
    T^*_1Q_1(A_K)T_1 &= Q_1(A_k) + \bint\limits_0^1 \frac{d}{d\lambda}\left(T^*_{\lambda}Q_1(A_k)T_{\lambda}\right)d\lambda\label{eq:25}\\
    &=Q_1(A_k) + \bint\limits_0^1 T^*_{\lambda}[\mathcal{K},Q_1(A_k)]T_{\lambda}d\lambda.\label{eq:26}
\end{align}
Then from the lemma above, we get
\begin{align}
    \eqref{eq:26}&= Q_1(A_k) + \bint\limits_0^1 T^*_{\lambda}(Q_2(\{K_k,A_k\}) + E_{Q_1}(K_k,A_k))T_{\lambda}d\lambda\label{eq:27}\\
    &= Q_1(A_k) + \bint\limits_0^1 T^*_{\lambda}(Q_2(\{K_k,A_k\})T_{\lambda}d\lambda + \bint\limits_0^1 T^*_{\lambda}E_{Q_1}(K_k,A_k))T_{\lambda}d\lambda\label{eq:28}
\end{align}
\begin{lemma}[Action of $T_\lambda$ on quadratic operators]\label{lem:4}
    
    \begin{align}
        T^*_{\lambda}Q_1(A_K)T_{\lambda} &=Q_1(A_k) + \bint\limits_0^{\lambda} T^*_{\lambda'}(Q_2(\{K_k,A_k\})T_{\lambda'}d\lambda' + \bint\limits_0^{\lambda} T^*_{\lambda'}E_{Q_1}(K_k,A_k))T_{\lambda'}d\lambda'\label{eq:29}\\
        T^*_{\lambda}Q_2(A_K)T_{\lambda} &= Q_2(A_k) + \bint\limits_0^{\lambda} T^*_{\lambda'}(Q_1(\{K_k,A_k\})T_{\lambda'}d\lambda' + \bint\limits_0^{\lambda} T^*_{\lambda'}E_{Q_2}(K_k,A_k))T_{\lambda'}d\lambda'\label{eq:30}
    \end{align}
\end{lemma}
\begin{proof}
    We prove the above by using the Duhamel's formula, redoing the above computation \eqref{eq:25} through \eqref{eq:28}.
\end{proof}
\begin{proposition}[Action of $T_1$ on $Q_1(A_k)$]

\begin{align}
    T^*_1Q_1(A_K)T_1 &= Q_1\left(\sum\limits_{m\geq0}\frac{\Theta^{2m}_K(A_k)}{(2m)!}\right) + Q_2\left(\sum\limits_{m\geq0}\frac{\Theta^{2m+1}_K(A_k)}{(2m+1)!}\right) \nonumber\\
        &+\bint\limits_0^1\bint\limits_0^\lambda\ldots \bint\limits_0^{\lambda_{n-1}} T^*_{\lambda_n}(Q_{\sigma(n)}(\Theta^n_K(A_k))T_{\lambda_n}d\lambda_n\ldots d\lambda_1 d\lambda + \sum\limits_{i=1}^{n}E_i
    \end{align}
    where $\sigma(n) = \begin{cases}
        1 &\text{for n even}\\
        2 &\text{for n odd}.
    \end{cases}$. 
\end{proposition}
\begin{proof}
    We have, from \eqref{eq:28},
    \begin{equation}
        T^*_1Q_1(A_K)T_1 = Q_1(A_k) + \bint\limits_0^1 T^*_{\lambda}(Q_2(\{K_k,A_k\})T_{\lambda}d\lambda + \bint\limits_0^1 T^*_{\lambda}E_{Q_1}(K_k,A_k))T_{\lambda}d\lambda\label{eq:32}
    \end{equation}
    We use \eqref{eq:30} from Lemma \ref{lem:4} above to arrive at
    \begin{align}
        \eqref{eq:32} &= Q_1(A_k) + \frac{1}{1!}Q_2(\{K_k,A_k\}) +\bint\limits_0^1\bint\limits_0^\lambda T^*_{\lambda_1}(Q_1(\{K_k,\{K_k,A_k\}\})T_{\lambda_1}d\lambda_1 d\lambda \nonumber\\&+ \bint\limits_0^1\bint\limits_0^\lambda T^*_{\lambda_1}(E_{Q_2}(K_k,\{K_k,A_k\})T_{\lambda_1}d\lambda_1 d\lambda +\bint\limits_0^1 T^*_{\lambda}E_{Q_1}(K_k,A_k))T_{\lambda}d\lambda\label{eq:33}.
    \end{align}
    Next we use \eqref{eq:29} from Lemma \ref{lem:4}
    \begin{align}
        \eqref{eq:33} &= Q_1(A_k) + \frac{1}{1!}Q_2(\{K_k,A_k\})+ \frac{1}{2!}Q_1(\{K_k,\{K_k,A_k\}\}) \nonumber\\
        &+\bint\limits_0^1\bint\limits_0^\lambda \bint\limits_0^{\lambda_1} T^*_{\lambda_2}(Q_2(\{K_k,\{K_k,\{K_k,A_k\}\}\})T_{\lambda_2}d\lambda_2d\lambda_1 d\lambda \nonumber\\
        &+\bint\limits_0^1\bint\limits_0^\lambda \bint\limits_0^{\lambda_1} T^*_{\lambda_2}E_{Q_1}(K_k,\{K_k,\{K_k,A_k\}\})T_{\lambda_2}d\lambda_2d\lambda_1 d\lambda \nonumber\\
        &+ \bint\limits_0^1\bint\limits_0^\lambda T^*_{\lambda_1}E_{Q_2}(K_k,\{K_k,A_k\})T_{\lambda_1}d\lambda_1 d\lambda +\bint\limits_0^1 T^*_{\lambda}E_{Q_1}(K_k,A_k)T_{\lambda}d\lambda\label{eq:34}.
    \end{align}
    For our convenience, we introduce the following notation for writing the nested anti-commutators 
    \begin{equation}
        \Theta^n_K(A_k) = \underbrace{\{K_k,\{\ldots,\{K_k}_\textrm{n times},A_k\}\ldots\}\}\\
    \end{equation}
    with
    \begin{equation}
        \Theta^0_K(A_k) = A_k.
    \end{equation}
    We also introduce another notation for the error terms 
    \begin{equation}
    E_n(K_k, A_k) =     
        \begin{cases}
        \bint\limits_{\Delta^{2n}}T^*_{\lambda}E_{Q_1}\left(\Theta^{2n}_{K}(A_k),K_k\right)T_{\lambda}d\lambda & \text{for n even}\\
        \bint\limits_{\Delta^{2n-1}}T^*_{\lambda}E_{Q_2}\left(\Theta^{2n-1}_{K}(A_k),K_k\right)T_{\lambda}d\lambda & \text{for n odd}
        \end{cases}.
    \end{equation}
    Then after multiple interations we arrive at
    \begin{align}
        T^*_1Q_1(A_K)T_1 &=  Q_1(\Theta^0_K(A_k)) + \frac{1}{1!}Q_2(\Theta^1_K(A_k))+ \frac{1}{2!}Q_1(\Theta^2_K(A_k)) \nonumber\\
        &+ \frac{1}{3!}Q_1(\Theta^3_K(A_k)) + \ldots \nonumber \\
        &+\bint\limits_0^1\bint\limits_0^\lambda\ldots \bint\limits_0^{\lambda_{n-1}} T^*_{\lambda_n}(Q_{\sigma(n)}(\Theta^n_K(A_k))T_{\lambda_n}d\lambda_n\ldots d\lambda_1 d\lambda \nonumber\\
        &+ E_n(K_k, A_k) + E_{n-1}(K_k, A_k)+\dots+E_1(K_k, A_k)\\
        &= Q_1\left(\sum\limits_{m\geq0}\frac{\Theta^{2m}_K(A_k)}{(2m)!}\right) + Q_2\left(\sum\limits_{m\geq0}\frac{\Theta^{2m+1}_K(A_k)}{(2m+1)!}\right) \nonumber\\
        &+\bint\limits_0^1\bint\limits_0^\lambda\ldots \bint\limits_0^{\lambda_{n-1}} T^*_{\lambda_n}(Q_{\sigma(n)}(\Theta^n_K(A_k))T_{\lambda_n}d\lambda_n\ldots d\lambda_1 d\lambda + \sum\limits_{i=1}^{n}E_i(K_k, A_k)
    \end{align}
    where $\sigma(n) = \begin{cases}
        1 &\text{for n even}\\
        2 &\text{for n odd}.
    \end{cases}$.
\end{proof}
%\section{This is an example for first level head---section head}\label{sec3}
%\subsection{This is an example for second level head---subsection head}\label{subsec2}
%\subsubsection{This is an example for third level head---subsubsection head}\label{subsubsec2}
%\section{Equations}\label{sec4}
%Notice the use of \verb+\nonumber+ in the align environment at the end of each line, except the last, so as not to produce equation numbers on lines where no equation numbers are required. The \verb+\label{}+ command should only be used at the last line of an align environment where \verb+\nonumber+ is not used.

%\begin{theorem}[Theorem subhead]\label{thm1}
%\end{theorem}

%\begin{proposition}
%\end{proposition}

%\begin{example}
%\end{example}

%\begin{remark}
%\end{remark}


%\begin{definition}[Definition sub head]
%\end{definition}

%\begin{proof}

%\end{proof}

%\begin{proof}[Proof of Theorem~{\upshape\ref{thm1}}]
%\end{proof}
\section{Error bounds}
\begin{definition}
    \begin{align}
        \Xi_\lambda(k) &\coloneq \expval{T_\lambda\Omega, a^*_ka_kT_\lambda\Omega}\\
        \Xi_\lambda &\coloneq \sup\limits_{k}\expval{T_\lambda\Omega, a^*_ka_kT_\lambda\Omega}
    \end{align}
\end{definition}
\begin{lemma}
    \begin{equation}
    \expval{T_\lambda\Omega, E_n(K_k,A_k)T_\lambda\Omega}\leq e^{||k||} \Xi_\lambda  
    \end{equation}
    
\end{lemma}

%\begin{appendices}

%\section{Section title of first appendix}\label{secA1}


%\end{appendices}


\bibliography{sn-bibliography}% common bib file
%% if required, the content of .bbl file can be included here once bbl is generated
%%\input sn-article.bbl


\end{document}
